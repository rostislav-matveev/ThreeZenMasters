\str{9}
\begin{ver}
Иккю появился на свет на восходе солнца первого дня 1394 года. Хотя,
по слухам, он был сыном молодого императора Го-Комацу (1377---1433),
при рождении Иккю был записан простолюдином.
Его мать, придворная дама и фаворитка Го-Коматсу, была
незаслуженно изгнана из дворца в результате
интриг ревнивой императрицы и её сторонников.
Таким образом, обстоятельства рождения Иккю были скромными, и
тем не менее в его первой биографии утверждается, что будучи ребёнком
он уже ``проявлял знаки дракона и носил отметины феникса.\comm{плохо}''
\end{ver}
\begin{ver}[1]
Иккю появился на свет на восходе солнца первого дня 1394 года. Хотя,
по слухам, он был сыном молодого императора Го-Комацу (1377---1433),
при рождении Иккю был записан простолюдином.
Его мать, придворная дама и фаворитка Го-Коматсу, была
незаслуженно изгнана из дворца в результате
интриг ревнивой императрицы и её сторонников.
Таким образом, обстоятельства рождения Иккю были скромными, и
тем не менее в его первой биографии утверждается, что ещё будучи ребёнком
он ``носил знаки дракона и отметины феникса.''\comm{плохо, что за
отметины такие?}
\end{ver}
\sep
\begin{ver}
В возрасте пяти лет Иккю был отправлен прислужником в Дзенский
храм Анкоку-джи в Киото. Там ему было обеспечено хорошее образование,
а так же защита от козней придворных и мнительных генералов. Последнее
было немаловажно в средневековой Японии, поскольку даже внебрачный сын
императора при благоприятных обстоятельствах и влиятельных
покровителях мог претендовать на престол.  В Анкоку-джи Иккю
/глубоко/основательно/ изучал буддийские тексты и классику Японии и
Китая.
\end{ver}
\begin{ver}[1]
В возрасте пяти лет Иккю был отправлен послушником в Дзенский
храм Анкоку-дзи в Киото. Там ему было обеспечено хорошее образование,
а так же защита от козней придворных и мнительных генералов. Последнее
было немаловажно в средневековой Японии, поскольку даже внебрачный сын
императора при благоприятных обстоятельствах и влиятельных
покровителях мог претендовать на престол.  В Анкоку-дзи Иккю
глубоко изучал буддийские тексты, а так же классику Японии и
Китая.
\end{ver}
\comm{acolyte -- прислужник,послушник???. Тут и далее}
\sep

\begin{ver}
Блестящий студент, чей талант был признан всеми, Иккю так же был
сообразительным и озорным ребёнком. Ниже приведены несколько анекдотов
из дней Иккю в качестве смышлёного прислужника.  
\end{ver}
\begin{ver}[1]
Блестящий студент, чей талант был признан всеми, Иккю так же был
сообразительным и озорным ребёнком. Ниже приведены несколько анекдотов
о смышлёном маленьком послушнике Иккю.
\end{ver}
\sep

\begin{ver}
Недолго после того как Иккю начал обучение, настоятиль
/велел/приказал/ ему загасить свечи /перед/на/ алтаре, прежде чем
готивиться ко сну\comm{см ориг}. Когда Иккю пришёл доложить о
выполненом задании, настоятель /поинтересовался/спросил/: ``Между
прочим, /как/каким образом/ ты загасил свечи?'' 

-- Задув, -- ответил Иккю.

-- Никогда /больше/ так не делай, -- разругался\comm{что-то вроде
  ``сделал выволочку''} настоятель. -- Будда
свят, а человеческое дыхание нечисто. Туши свечи, махая рукой или
веером.

На следующий день, когда настоятель вошёл в главную залу для утренней
службы, он обнаружил Иккю /повторяющим мантры/\orig{chanting} сидя
спиной к алтарю.

-- Ты что делаешь, маленький глупец? -- взорвался /он/настоятель/.

-- Вы сказали мне, что дыхание нечисто и не должно быть направлено на
образ Будды. Разве возможно повторять мантры не дыша?

-- Это совсем другое. -- пробормотал /сконфуженый/в замешательстве/
настоятель, и велел Иккю развернуться.
\end{ver} 
\begin{ver}[1]
Вскоре после того как Иккю начал обучение, настоятель
велел ему загасить свечи перед алтарём, прежде чем
готивиться ко сну. Когда Иккю пришёл доложить о
выполненом задании, настоятель поинтересовался: ``Между
прочим, каким способом ты загасил свечи?'' 

-- Задув, -- ответил Иккю.

-- Никогда больше так не делай, -- разругался настоятель. -- Будда
   свят, а человеческое дыхание нечисто. Туши свечи, махая рукой или
   веером.

На следующий день, когда настоятель вошёл в главную залу для утренней
службы, он обнаружил Иккю /повторяющим мантры/\orig{chanting} сидя
спиной к алтарю.

-- Ты что делаешь, маленький глупец? -- взорвался настоятель.

-- Вы сказали мне, что дыхание нечисто и не должно быть направлено на
   образ Будды. Разве возможно повторять мантры не дыша?

-- Это совсем другое. -- пробормотал настоятель в замешательстве, и
   велел Иккю развернуться.
\end{ver}
\comm{chant -- ???. Тут и далее.}
\sep

\str{10}
\begin{ver}
Иккю однако не угомонился\orig{was irrepressible}. Настоятель очень
любил /определённые/\orig{certain type of} конфеты, и /прятал/хранил/
лыбимые сладости в своей комнте спратанными в горшке, предупреждая
Иккю и остальных прислужников: ``Эти конфеты
полезны\orig{beneficial} взрослым, а если ребёнок съест такую конфету,
его ждёт мгновенная смерть.''

Иккю /не был одурачен/???/ ни на /мнгновение/секунду/ и как только
настоятеля не было поблизости он опустошил содержимое горшка
поделившись с друзьями. После этого, он [намеренно] разбил чайную
чашку в комнате настоятеля. По возвращении /тот/настоятель/ нашёл Иккю в
слезах.
``Убираясь в Вашей комнате, -- убедительно всхлипывал Иккю, -- я
случайно разбил эту драгоценную чашу. Чтобы искупить мою ужасную
ошибку, я проглотил ядовитую конфету. Ничего не произошло, так что я
съел и остальные, будучи уверен что мне придёт конец, как Вы и
говорили. Но, вот, я не умер. Простите меня, учитель!''. 
\comm{Не очень близко к тексту.}
\end{ver}

\begin{ver}[1]
Иккю однако не угомонился. Настоятель очень
любил определённый сорт конфет, и прятал
любимые сладости в своей комнате спратанными в горшке, предупреждая
Иккю и остальных послушников: ``Эти конфеты
полезны взрослым, а если ребёнок съест такую конфету,
его ждёт мгновенная смерть.''

Иккю не был одурачен ни на секунду и как только
настоятеля не было поблизости, он опустошил содержимое горшка,
поделившись с друзьями. После этого, он намеренно разбил чайную
чашку в комнате настоятеля. По возвращении настоятель нашёл Иккю в
слезах.
``Убираясь в Вашей комнате, -- убедительно всхлипывал Иккю, -- я
случайно разбил эту драгоценную чашу. Чтобы искупить мою ужасную
ошибку, я проглотил ядовитую конфету. Ничего не произошло, так что я
съел и остальные, будучи уверен что тогда мне придёт конец, как Вы и
говорили. Но, вот, я всё-таки не умер. Простите меня, учитель!''
\end{ver}
\sep
\begin{ver}
В другой раз, другой прислужник, во время уборки, случайно разбил любимую чашу
настоятеля. Страшась настоятельского гнева, прислужник
/взмолился/попросил/ находчивого Иккю выручить его из переделки\orig{to
  get him out of jam}. ``Предоставь это мне,'' -- успокоил его Иккю.
Когда настоятель вернулся, Иккю встретил его у ворот.

-- Учитель\orig{master}, -- подступился\orig{said softly} к нему
Иккю, -- Вы часто наставляли нас, что всё что было рождено, умрёт, и
всё, что обладает материальной формой,
исчезнет\orig{perish=погибнет}.

-- Это так, -- отозвался настоятель. -- Таковы неизбежные
законы\orig{realities=реалии} жизни.

-- Учитель, у меня плохие новости, -- сказал /грустный/грустно/ Иккю.
-- Вашей любимой чаше /пришло время/настал час/ /умереть/''
\end{ver}

\begin{ver}[1]
В другой раз, другой прислужник, во время уборки, случайно разбил любимую чашу
настоятеля. Страшась настоятельского гнева, прислужник
взмолился, чтобы находчивый Иккю выручил его из переделки.
``Предоставь это мне,'' -- успокоил его Иккю.
Когда настоятель вернулся, Иккю встретил его у ворот.

-- Учитель, -- подступился к нему
Иккю, -- Вы часто наставляли нас, что всё что было рождено, умрёт, и
всё, что обладает материальной формой,
исчезнет.

-- Это так, -- отозвался настоятель. -- Таковы неизбежные
законы жизни.

-- Учитель, у меня плохие новости, -- грустно сказал Иккю.
-- Вашей любимой чаше пробил час.
\end{ver}
\sep

\str{11}
\begin{ver}
Репутация маленького смышлёного монаха достигла ушей сёгуна
Ёшимитсу, и Иккю был призван в замок для встречи /с сёгуном/.

-- Я слышал, что ты очень сообразителен, -- обратился к Иккю
сёгун. -- Как ты считаешь, ты можешь изловить тигра?

-- Да, Господин, уверен, что смогу, -- самоуверенно заявил
Иккю.\comm{повторение}

-- Вот тебе верёвка. Поймай-ка этого, -- /бросил
вызов/\orig{challenged} сёгун, показав на тигра нарисованного на
большом /сёдзи/\orig{screen}.

Иккю, без колебания, встал напротив сёдзи, и приготовив верёвку,
закричал: ``Теперь, Господин, гоните тигра на меня!''
\end{ver}

\begin{ver}[1]
Репутация маленького смышлёного монаха достигла ушей сёгуна
Ёсимицу, и Иккю был призван в замок для встречи с сёгуном.

-- Я слышал, что ты очень сообразителен, -- обратился к Иккю сёгун. --
   Как ты считаешь, ты можешь изловить тигра?

-- Да, Господин, думаю, что смогу, -- самоуверенно заявил Иккю.

-- Вот тебе верёвка. Поймай-ка этого, -- бросил вызов сёгун, показав
   на тигра нарисованного на большом сёдзи.

Иккю, без колебания, встал напротив сёдзи, и приготовив верёвку,
закричал: ``Теперь, Господин, гоните тигра на меня!''
\end{ver}
\sep

\begin{ver}
Один из покровителей храма любил\orig{was fond of} кожанную одежду,
что, с точки зрения буддизма, не являлось подходящим одеянием /для
адепта/. 

Когда он как-то раз подходил к храму, его /приветствовал/\orig{was
  greeted, в пассивном залоге} /знак/надпись/???/ на табличке\orig{on
  board} говорящяя ``Кожанные вещи /недопустимы/не разрешены/ в
пределах храма. Нарушивший это правило будет изрядно избит.''
Распознав каллиграфию проказливого\orig{impish} Иккю, он ворвался в храм,
кипя негодованием на маленького монаха. ``А как же большой барабан в
главном зале? -- поинтересовался\orig{demanded to know=потребовал
  узнать} покровитель. -- Разве он не из кожи?''

``Да, это так, -- ответил Иккю, и продолжил: -- Вот мы и бьём его с
утра до вечера. Если Вы настаиваите на том, чтобы носить кожу, то и с
Вами поступят точно так же.''   
\end{ver}

\begin{ver}[1]
Один из покровителей храма любил носить кожанную одежду,
что, с точки зрения буддизма, не являлось подходящим одеянием для
адепта. 

Когда он как-то раз подходил к храму, у входа его приветствовал знак:
``Кожанные вещи не разрешены в пределах храма. Нарушивший это правило
будет изрядно избит.''  Распознав каллиграфию проказливого Иккю, он
ворвался в храм, кипя негодованием на маленького монаха. ``А как же
большой барабан в главном зале? -- стал протестовать покровитель. --
Разве он не из кожи?''

``Да, это так, -- ответил Иккю, и продолжил: -- Вот мы и бьём его с
утра до вечера. Если Вы настаиваите на том, чтобы носить кожу, то и с
Вами поступят точно так же.''
\end{ver}
\sep

\str{12}
\begin{ver}
Пытаясь /отомстить/отыграться/\orig{turn the tables}, /этот/ покровитель
пригласил настоятеля и Иккю на пир из вегетарианских блюд. Однако, в
миске у Иккю лежал кусок рыбы. Иккю сразу жадно проглотил его.

-- Минуту! -- стал упрекать Иккю покровитель. -- Разве ты не знаешь,
что буддийским монахам недозволено употреблять в пищу плоть?

-- Мой рот как тракт Камакура, -- сообщил ему Иккю. -- /высокое и
низменное/высокие и низкие/, мясники и зеленщики, все проходят по нему
беспрепятственно.

Непременно решив /отомстить/восторжествовать/\orig{to get the better}
над ???\orig{smart aleck=умная мелкая селёдка, типа умный паршивец},
покровитель вытащил свой меч.

-- А /как это/это как/ пройдёт? -- спросил он угрожающе.

-- Друг или враг? -- поинтересовался Иккю.

-- Враг! -- воскликнул патрон.

-- Враги не пройдут! -- /вызывающе/демонстративно/ отвечал Иккю.

-- Тогда друг! -- закричал покровитель.

Иккю закашлялся, будто подавившись: ``Извините, но ворота только что закрылись.''
\end{ver}
\begin{ver}[1]
Пытаясь отыграться, этот покровитель
пригласил настоятеля и Иккю на пир из вегетарианских блюд. Однако, в
миске у Иккю лежал кусок рыбы. Иккю сразу жадно проглотил его.

-- Минуту! -- стал упрекать Иккю покровитель. -- Разве ты не знаешь,
что буддийским монахам недозволено употреблять в пищу плоть?

-- Мой рот как тракт Камакура, -- сообщил ему Иккю. -- высокое и
низкое, мясники и зеленщики, все проходят по нему
беспрепятственно.

Непременно решив восторжествовать
над ???\orig{smart aleck=умная мелкая селёдка, типа умный паршивец},
покровитель вытащил свой меч.

-- Ну а это как пройдёт? -- спросил он угрожающе.

-- Друг или враг? -- поинтересовался Иккю.

-- Враг! -- воскликнул патрон.

-- Враги не пройдут! -- демонстративно отвечал Иккю.

-- Тогда друг! -- закричал покровитель.

Иккю закашлялся, будто подавившись: ``Извините, но ворота только что
закрылись.''

\comm{smart aleck!!!!}
\end{ver}
\sep

\begin{ver}
В похожей истории Иккю застал заболевшего\orig{long suffering}
настоятеля, поедавшим похлёбку из рыбы.

-- Учитель, вы утверждали, что плоть в пищу абсолютно запрещена в
пределах храма. Но раз уж теперь это в порядке [вещей], -- сказал Иккю
уставившись в миску, -- то и я бы попробовал.

-- Нет, нет! -- настаивал\comm{в оригинале прошедшее} настоятель. --
Это и в самом деле большой грех для
послушников\orig{novice=новичок,начинающий} есть мясо. В моём же
случае... Я просто совершаю погребальный обряд для этой рыбы.

-- Как так? -- поинтересовался Иккю.

-- Я сказал этой рыбе: ``Рыба, теперь ты как высохшая ветвь. Если бы я
и отпустил тебя, тебе уже не вернуться в воду. Но если ты послужишь мне
лекарством, то может быть скоро станешь Буддой!''\comm{не очень близко
к тексту} 

На следуюший день Иккю со всех ног помчался на рынок и купил там
живого карпа. Только он собирался разрезать рыбу, как вошёл
настоятель.

-- Ты что /творишь/делаешь/! Убить живое в десять раз хуже, чем
съесть! -- воскликнул он.

-- Нет нет, послушайте, -- сказал Иккю держа в руках извивающуюся
рыбёшку. -- Рыба, ты как живая ветвь. Если я отпущу тебя, ты
ускользнёшь, /а/и/ я останусь голодным. Чем тебе плавать в в воде, лучше
доставь мне удовольствие в качестве обеда\comm{не близко к ориг}.

С этими словами он отрубил рыбе голову и кинул в котел\comm{не близко
  к ориг}.

/Снова/ещё раз/ посрамленный настоятель не нашёлся что сказать.
Такая /шоковая/эпатажная/ тактика стала характерной для
Иккю\comm{плохо, см ориг}
\end{ver}

\begin{ver}[1]
В похожей истории Иккю застал заболевшего настоятеля, поедавшим
похлёбку из рыбы.

-- Учитель, вы утверждали, что плоть в пищу абсолютно запрещена в
   пределах храма. Но раз уж теперь это в порядке вещей, -- сказал
   Иккю уставившись на миску, -- то и я бы попробовал.

-- Нет, нет! -- возразил настоятель. -- Это и в самом деле большой
   грех для послушников есть мясо. В моём же случае... Я просто
   совершаю погребальный обряд для этой рыбы.

-- Как так? -- поинтересовался Иккю.

-- Я сказал этой рыбе: ``Рыба, теперь ты как засохшая ветвь. Если бы я
   и отпустил тебя, тебе уже не вернуться в воду. Но если ты послужишь
   мне лекарством, то может быть скоро станешь Буддой!''

На следуюший день Иккю со всех ног помчался на рынок и купил там
живого карпа. Вернувшись, он уже собирался разрезать рыбу, когда вошёл
настоятель.

-- Что ты творишь! Убить живое в десять раз хуже, чем съесть! --
   воскликнул он.

-- Нет, нет, послушайте, -- сказал Иккю держа в руках извивающуюся
   рыбёшку. -- Рыба, ты как живая ветвь. Если я отпущу тебя, ты
   ускользнёшь, и я останусь голодным. Чем тебе играть в воде,
   лучше доставь мне удовольствие в качестве обеда.

С этими словами он отрубил рыбе голову и кинул в котел.

Снова посрамленный настоятель не нашёлся что сказать.  Такая шоковая
тактика стала характерной для Иккю.
\end{ver}
\sep

\begin{ver}
Своё первое стихотворение по Китайскому канону, что являлось очень
/требовательной/взыскательной/ формой, Иккю сочинил, когда ему было
двенадцать. 
\end{ver}
\begin{ver}[1]
Своё первое стихотворение по Китайскому канону, что является очень
взыскательной формой, Иккю сочинил, когда ему было
двенадцать. 
\end{ver}
\sep
\begin{ver}
\begin{verses}
  \centerline{``/Сухая/высохшая/пожухлая/ Трава /у/около/
    /Поместья/\orig{manor=феодальное поместье}''}
В запустенье осени, поёт /забытая/покинутая/ красавица,\\
Не идёт /вестник/посланник/гонец/посыльный чтобы позвать её снова в
поместье,\\
Слава и позор, радость и /несчастье/грусть/печаль/ -- она видела всё это,\\
/Благосклонность/расположение/покровительство/интерес/ её господина
была мелкой, а трава /пренебрежения/небрежения/ глубокая.
\end{verses}
\end{ver}

\begin{ver}[1]
\begin{verses}
  \centerline{``Сухая Трава у Замка''}
  В осеннем запустенье, поёт забытая краса,\\
  Нет вестника её в замок вернуть,\\
  Позор и слава, радость и несчастье -- знакомо всё ей,\\
  Как мелка благосклонность господина, так высока забвения трава.\\
\end{verses}
\end{ver}

\begin{ver}[1]
\begin{verses}
  \centerline{``Сухая Трава у Замка''}
  Унылой осенью, поёт забытая краса,\\
  Гонца не шлют её в замок позвать,\\
  Позор и слава, радость и несчастье -- знакомо это всё,\\
  Сквозная благосклонность господина, густая забвения трава.\\
\end{verses}
\end{ver}

\begin{ver}[1]
\begin{verses}
  \centerline{``Сухая Трава у Замка''}
  Унылой осенью, поёт забытая краса,\\
  Гонца не шлют её в замок позвать,\\
  Позор и слава, радость и несчастье -- знакомо это всё,\\
  Была ничтожной милость господина, но велика забвения трава.\\
\end{verses}
\end{ver}

\begin{ver}[1]
\begin{verses}
  \centerline{``Сухая Трава у Замка''}
  В осеннем запустенье, поёт забытая краса,\\
  Гонца не шлют её в замок вернуть,\\
  Позор и слава, радость и несчастье -- знакомо всё,\\
  Была ничтожной милость господина, но высока забвения трава.\\
\end{verses}
\end{ver}

\begin{ver}[1]
\begin{verses}
  \centerline{``Сухая Трава у Замка''}
  В осеннем запустенье, поёт забытая краса,\\
  Гонца не шлют её в замок вернуть,\\
  Позор и слава, радость и печаль ей знакомы,\\
  Была ничтожной милость господина, но высока забвения трава.\\
\end{verses}
\end{ver}

\begin{ver}[1]
\begin{verses}
  \centerline{``Сухая Трава у Замка''}
  В осеннем запустенье, поёт забытая краса,\\
  Не шлют гонца её в замок вернуть,\\
  Позор и слава, радость и печаль ей знакомы,\\
  Как мелка благосклонность господина, так высока забвения трава.\\
\end{verses}
\end{ver}

\begin{ver}[2]
\begin{verses}
  \centerline{``Сухая Трава у Замка''}
  В осеннем запустенье, поёт покинутая красавица,\\
  Не шлют гонца её в замок вернуть,\\
  Ей слава и позор, радость и печаль знакомы,\\
  Мелкой господина была милость, но высока забвения трава.\\
\end{verses}
\end{ver}

\begin{ver}[3]
\begin{verses}
  \centerline{``Сухая Трава у Замка''}
  В осеннем запустенье, поёт забытая краса,\\
  Не шлют гонца её в замок вернуть,\\
  Ей слава и позор и радость и печаль знакомы.\\
  Как мелка благосклонность господина, так высока пренебрежения трава.
\end{verses}
\end{ver}
\sep

\str{13}
\begin{ver}
Хотя ребёнком Иккю был полон игривости, его характер всегда имел и
серьёзную сторону. Этот стих пропитан Буддистким пессимизмом, жалобой
на непостоянство вещей, и он отражает глубокое чувство обиды Иккю на
то, как поступили с его матерью.  Иккю был особенно близок со своей
матерью и он так и не смог преодолеть горечь связанную с её
незаслуженным изгнанием от двора.
\comm{Это противоречит другим источникам и тексту ниже}  
В другом стихотворении, сочинённом
позднее, Иккю /выражает своё раздражение/\orig{frets}:
\end{ver}
\begin{ver}[1]
Хотя ребёнком Иккю был полон игривости, его характер всегда имел и
серьёзную сторону. Этот стих пропитан Буддистким пессимизмом, жалобой
на непостоянство вещей, и он отражает глубокое чувство обиды Иккю на
то, как поступили с его матерью.  Иккю был особенно близок со своей
матерью и он так и не смог преодолеть горечь вызванную  её
незаслуженным изгнанием от двора.
\comm{Это противоречит другим источникам и тексту ниже. Может ``\ldots
не сразу смог преодолеть горечь вызванную её
незаслуженным изгнанием от двора''}  
В другом стихотворении, сочинённом
позднее, Иккю так выражает своё негодование:
\end{ver}
\sep

\begin{ver}
  \begin{verses}
    Такая /изысканная/утонченная/изящная/совершенная/
    /красота/краса/красавица/\\
    \rule{7mm}{0mm}нарумяненая и напудренная,\\
    /Даже/Сам/ Будда не мог бы /устоять перед н/сопротивляться/ ей;\\
    Она имеет душу Китайской Нефритовой Красавицы,\\
    Однако /это/именно/ в Японии она теперь
    /чахнет/томится/\\
    \rule{7mm}{0mm}изнывает/вянет/тоскует/. 
  \end{verses}
\end{ver}

\begin{ver}[1]
  \begin{verses}
    На ней румяна и пудра -- такая совершенная красавица,\\
    Даже сам Будда не смог бы устоять перед ней;\\
    Воплошение Китайской Нефритовой Красавицы,\\
    В Японии она теперь тоскует. 
  \end{verses}
\end{ver}

\begin{ver}[1]
  \begin{verses}
    Так румяна и бела -- совершенная красавица,\\
    Даже сам Будда не смог бы устоять перед ней;\\
    Воплошение самой Китайской Нефритовой Девы,\\
    В Японии она теперь тоскует. 
  \end{verses}
\comm{Немножко плагиат}
\end{ver}

\begin{ver}[1]
  \begin{verses}
    Так румяна и бела -- совершенная красавица,\\
    Даже сам Будда не смог бы устоять перед ней;\\
    Воплошение самой Китайской Нефритовой Девы,\\
    Но в Японии она теперь тоскует. 
  \end{verses}
\end{ver}

\begin{ver}[2]
  \begin{verses}\it
    Такая белая и румяная -- совершенная красавица,\\
    Даже сам Будда не смог бы устоять перед ней;\\
    Воплошение самой Китайской Нефритовой Девы,\\
    Но вот в Японии она теперь тоскует. 
  \end{verses}
\end{ver}

\begin{ver}[2]
  \begin{verses}
    Такая изысканая красавица -- белая и румяная,\\
    Даже сам Будда не смог бы устоять перед ней;\\
    Воплошение самой Китайской Нефритовой Девы,\\
    Но вот в Японии она теперь тоскует. 
  \end{verses}
\end{ver}
\sep

\begin{ver}
В добавок к осуждению несправедливого обрашения с его матью, эти две
/поэмы/стихотворения/ /раскрывают/указывают/обращают внимание/ на ещё
одну /озабоченность/ Иккю: частенько
/беспокойные/затруднёные/сложные/проблематичные/ отношения между
мужчиной и женщиной. В условиях /вырождения/декаданса/ времён скорее
всего Иккю был рано познакомлен с гомосексуальной любовью в стенах его
храма. Иккю не заинтересовали подобного рода отношения, но его
/очарование/увлечение/ противоположным поломо сталось с ним до
последнего дня его восмидесятисемилетней жизни.
\comm{см ориг. Много неточностей. Плохой перевод}
\end{ver}
\begin{ver}[1]
Кроме осуждения несправедливого обращения с матерью, эти два
стихотворения указывают на ещё одну тему беспокоившую Иккю: часто
сложные отношения между мужчиной и женщиной. В условиях декаданса того
времени, скорее всего Иккю был рано познакомлен с гомосексуальной
любовью в стенах его храма. Подобного рода отношения не заинтересовали
Иккю, но его очарование противоположным полом осталось с ним до
последнего дня его восмидесятисемилетней жизни.  
\comm{Плохо}
\end{ver}
\sep

\str{14}
\begin{ver}
В 1410, /испытывая отвращение к  жадности и
/продажности/разложению/испорченности/извращённости/\orig{corruption}
старших монахов шеснадцатилетний Иккю /покидает/убегает/ из
Анкоку-джи. О своём /уходе/отбытии/ он заявил следующим стихотворением
\end{ver}
\begin{ver}[1]
В 1410 году, испытывая отвращение к  жадности и
продажности 
старших монахов, шеснадцатилетний Иккю покидает 
Анкоку-джи. О своём уходе он заявил следующим стихотворением:
\end{ver}
\sep

\begin{ver}
\begin{verses}
  Заполненый стыдом, я едва могу держать язык за зубами,\\
  Слова Дзен /повержены/сокрушены/, и демонические силы выходят
  победителями.\\
  Эти монахи должны учить Дзен'у,\\
  Но всё что они делают это хвастаются семейной историей.
\end{verses}
\end{ver}
 
\begin{ver}[1]
\begin{verses}
  Полон стыда, я едва могу держать язык за зубами,\\
  Дзен сокрушён, и демонические силы торжествуют победу.\\
  Те монахи, что должны демострировать Дзен,\\
  Лишь гордятся своей родословной.
\end{verses}
\end{ver}

\begin{ver}[1]
\begin{verses}
  Полон стыда, я едва могу держать язык за зубами,\\
  Дзен сокрушён, и демонические силы торжествуют победу.\\
  Mонахи должны демострировать Дзен,\\
  А они лишь гордятся своей родословной.
\end{verses}
\end{ver}
\sep   

\begin{ver}
Иккю продолжил /учебу/тренировку/ /под руководством/ Кен'о (--1414),
эксцентричным старым монахом, живущим в удалённой
/полуразрушенной/ветхой/ /лачуге/хижине/ в холмах неподалёку от Киото.
/Раньше/в былые годы/в молодости/\orig{earlier in his life} Кен'о
/вызвал/был причиной/ переполоха, отказавшись принять инка от своего
учителя Муина, тогдашнего настоятеля Миёшин-джи\comm{посмотреть
  принятую транслитерацию}. В те времена подобные сертификаты --
частенько купленные или полученные жульническим образом -- были
необходимы для получения места в крупном храме. Таким образом,
/мятежный/бунтовщический/ поступок Кен'о лишил Кен'о возможности стать
членом меинстримовского Буддисткого сообщества, что вполне
удовлетворяло его самого и его единственного ученика, упрямого и
/непреклонного/решительного/ Иккю.
\end{ver}

\begin{ver}[1]
Иккю продолжил /учебу/тренировку/ /под руководством/ Кен'о (--1414),
эксцентричным старым монахом, живущим в удалённой
/полуразрушенной/ветхой/ /лачуге/хижине/ в холмах неподалёку от Киото.
/Раньше/в былые годы/в молодости/\orig{earlier in his life} Кен'о
/вызвал/был причиной/ переполоха, отказавшись принять инка от своего
учителя Муина, тогдашнего настоятеля Миёшин-джи\comm{посмотреть
  принятую транслитерацию}. В те времена подобные сертификаты --
частенько купленные или полученные жульническим образом -- были
необходимы для получения места в крупном храме. Таким образом,
/мятежный/бунтовщический/ поступок Кен'о лишил Кен'о возможности стать
членом меинстримовского Буддисткого сообщества, что вполне
удовлетворяло его самого и его единственного ученика, упрямого и
/непреклонного/решительного/ Иккю.

\comm{Not done}
\end{ver}
\sep

\begin{ver}
В то же время Иккю изучал литературу с учёным монахом по имени
Сейсо\comm{посмотреть транслитерацию}. Однажды сёгун
Ешимочи(1386---1468) появился у входа в храм Сейсо и потребовал
/посмотреть/\orig{examine} один
из свитков находящихся в распоряжении храма. Тогда как другие ученики
/съёжились по углам/сгрудились в углу/, Иккю /смело/\orig{boldly}
принёс свиток, но не спустился ко входу, чтобы подать свиток сёгуну
или одному из его слуг, как того требовал этикет. В конце концов,
сёгун, пути которого уже пересекались с Иккю, поднялся и сам взыл
свиток у Иккю. Посмотрев свиток, сёгун вернул его и удалился, не
сказав ни слова. Позднее, молодой Иккю, /заслужил похвалу/ за смелость
перед лицом /властной/ светской власти. 
\end{ver}

