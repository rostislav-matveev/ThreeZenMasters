\str{14}
\begin{ver}
  Иккю оставался с Кен'о до самого дня смерти этого бескомпромисного
  монаха в 1414ом году. После того как Иккю /послужил/стал/ копателем
  могилы/могильщиком/ и единственным оплакивателем своего учителя,
  /обезумевший/очень сильно расстроенный/ Иккю бесцельно бродил вокруг
  Киото, /сетуя на/оплакивая/ свою потерю и пытаясь унять боль
  повторяя\orig{chanting=распевая бормоча; то что делают с мантрами,
    молитвой} и медитируя.  В течение этого
  /мрачного/унылого/бесцветного/ периода он несколько раз он навестил
  свою мать, которая была так обеспокоена его
  /всклокоченным/растрёпанным/\orig{disheveled} видом, и
  /спутанным/неуравновешенным/сбитым/ состоянием /души/ума/, что
  велела одному из слуг тайно следовать за Иккю во время его скитаний.

  Совершенно подавленный Иккю вознамерился броситься в озеро Бива,
  сказав: ``Раз уж я совершенно бесполезен здесь в этом мире\orig{here on
    earth}, то пусть я хоть послужу кормом для рыб!\comm{см ориг}''.
  К счастью, слуга посланный его матерью оказался поблизости и
  предотвратил самоубийство. После того как Иккю успокоился, слуга
  передал ему письмо от его матери, в котором она просила его
  оставаться в живых ради неё. \comm{Письмо существует и я читал его в
  другом источнике. Оно очень хорошее и я бы взял на себя смелость его
  привести здесь, хотя в оригинале только одна цитата.}
  Также она писала: ``Ты достигнешь просветления; будь настойчив.''
  Иккю согласился вернуться в дом матери и обдумать свою дальнейшую судьбу.
\end{ver}
