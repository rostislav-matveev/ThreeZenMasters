\str{15}
\begin{ver}
  Он решил добиваться ученичества у Касо (1352---1428), Дзен мастера,
  который считался ещё более /суровым/требовательным/, чем Кен'о. Оба
  строгих наставника отлично представляли Дзен
  /требовательной/взыскательной/ традиции Дайто. Дайто (Шухо Миёхо,
  1282---1337), после достижения просветления под руководством Даиё,
  (Нанпо Ёмиё, 1235---1308), проводил время, углубляя своё
  /понимание/инсайт/осознание/ живя /нищим/бродягой/ в окрестностях Киотского
  моста пятой авеню.  Местные /бандиты/разбойники/ частенько проверяли
  остроту своих ножей на беспомощных бродягах, но когда однажды ночью
  стали угрожать /Дайто/Дайтё/, он встретил угрозы с бесстрашным
  спокойствием. 
  \begin{verse}\it
    Мои /испытания/искушения/ /продолжаются/возобновляются/\\
    всё /дальше и дальше/снова и снова/ --\\
    Теперь я увижу\\
    насколько /уравновешен/постоянен/непреклонен/спокоен/.../
    мой ум на самом деле    
  \end{verse}
Обезоруженные непоколебимым присутсвием духа Дайтё, головорезы его
больше никогда не беспокоили.
\comm{Посмотреть все транслитерации. Посмотреть переводы терминов,
  например realization в других переводах Дзенской
  литературы. Посмотреть название моста.}

Даже после того, как в 1324 году император Ханазоно\comm{транслит}
назначил/установил Дайтё в качестве/ первого настоятеля храма
Дайтоку-джи, этот монах продолжал поддерживать строгую дисциплину, как
для себя, так и для своих учеников. До сего дня ``Последние
наставления'' декламируются в Дайтоку-джи.

\begin{quote}\it
  Все вы, пришедшие в этот горный монастырь, не забывайте, что вы
  здесь ради Пути, а не ради одежды и еды... Напоминайте себе и днём и
  ночью познавать /непозноваемое/невыразимое/. От начала и до конца,
  /исследуйте/интересуйтесь/проникайте/ во бсе мельчайшие
  детали. Время летит как стрела, не тратите энергию на
  ерунду\orig{trivial matters}. Будте внимательны! Будте внимательны!

  После того, как этот старый монах закончит свой путь, некоторые из
  вас /будут руководить/\orig{preside} великими храмами с
  великолепными зданиями и огромными библиотеками, украшенными золотом
  и серебром [и у них] будет много последователей.
  Другие посвятят себя изучению сутр, мантрам\orig{chants=песнопение},
  или постоянной медитации и строгому следованию
  /заповедям/заветам/правилам/. Каким образом вы бы не действовали,
  если ум не нацелен на чудесное, на трансцендентный путь Будд и
  Патриархов, вы пренебрегаете законом причинности, и сокрушаете
  учение.
  Такие люди являются бесами\orig{devils} и никогда не будут моими
  [духовными] наследниками. Тот, кто /смотрит внутрь себя/заботится о
  своём внутреннем состоянии/.../, даже если он живёт в /хижине/лачуге/
  в далёкой провинции\orig{countryside=антоним к ``город''}, питаясь
  дикими кореньями\orig{vegetables}, приготовленными в
  дырявом\orig{battered=изношенный} /котле/горшке/, [такой человек]
  /видит/встречает/получает/\orig{encounter} мою традицию /ежедневно/всё
  времы/ежечасно/постоянно/, и получает моё учение с благодарностью.
  Разве можно не принимать этого всерьёз? Старайтесь! Старайтесь!
  \comm{Уверен, переводы этого текста существуют. Посмотреть.}
\end{quote}

В день своей смерти, Дайтё, который был хром на одну ногу, обратился к
своей /деформированной/ конечности [с такими словами]: ``Всю свою
жизнь я следовал за тобой, но сегодня [наконец] ты пойдёшь за
мной!'' Сказав /это/так/, он /привел/ свою ногу в правильную позицию для
медитации, сломав кость и проткнув ей кожу. 
\end{ver}
