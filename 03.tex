\str{16}

\begin{ver}
Касо недолгое время тоже был настоятелем Дайтоку-джи, хотя он гораздо
больше предпочитал своё маленькое убежище на озере Бива. В отличие от
/богатства/изобилия/ и фривольности Дайтоку-джи, режим в
/убежище/\orig{retreat} Касо было испытанием характера даже для самых
его  ревностных учеников. 

Когда Иккю появился там, ему было /совершенно однозначно/ отказано в
приёме. Иккю /упорствовал/не сдавался/, усевшись и ворот и проведя там
пять дней пока /оскорбления/ не перешли в телесные,
сторож\orig{gatekeeper} окатывал его помоями и лупил его метлой. /Не
оттолкнутый/\orig{undeterred} Иккю подвергался такому оскорбительному
обрашению почти неделю, прежде чем Касо смягчился и впустил его.

Как член /коммьюнити/общины/ Касо Иккю /подвергся/ режиму в полной
мере: тяжелый труд, скудная еда, короткий сон и долгие периоды
медитации.   

Благодаря своим художественным способностям, Иккю получал\orig{gained}
некоторую передышку в изнурительном режиме, работая в Киото где он
раскрашовал кукол своими собственными рисунками на кимоно\comm{см
ориг}, и делал пакетики для /.../\orig{=incense}. Заработанные деньги
помогали поддерживать /убежище/пристанище/\orig{retreat} Касо. 
\end{ver}

