\str{17}
\begin{ver}
  Иккю изучал различные коаны /с/под руководством/ Касо, но в
  особенное недоумение его повергал пятнадцатый из ``Дверь без Двери''
  /Мумокана/Мумона/
  \begin{quote}\it
    Когда Тунь-шань пришёл учиться у Юнь-мэна, мастер спросил монаха:
    
    -- Откуда ты?

    -- Из округа Ч'а, -- овечал Тунь-шань.

    -- Где ты обучался этим летом? -- поинтересовался Юнь-мэн.

    -- В Даосистком монастыре в /Хунане/Хонане/.

    -- Когда же ты ушёл оттуда?

    -- Двадцать пятого августа.

    -- Ты заслужил шестьдесят ударов моей /палки/посоха/! --
    /прогромыхал/стал грозиться/ Юнь-мэн [он не привёл свою угрозу в
      исполнение]. 

    На следующий день Тунь-шань вернулся к Юнь-мэну и спросил: ``Вчера
    Вы сказали, что я заслужил шестьдесят ударов Вашей палки. В чём
    моя ошибка?''

    -- Ты бесполезный мешок с рисом! Что ты /шляешься/слоняешься/ туда и сюда?!

    В этот момент Тунь-шаня настигло /просветление/прозрение/.
  \end{quote}
\end{ver}

\begin{ver}
  Однажды, когда Иккю был в городе, он услышал, в
  исполнении слепого министреля, /рассказ/балладу/ о Леди Гио,
  фаворитки генерала Тайра-но-Киёмори (--1181) эры Хейан, отвергнутой,
  когда генерал увлёкся другой красавицей -- Хотокэ (Будда). В конце
  концов обе девушки, уставшие от мира и его скорбей, ушли в
  монахини. По причинам, ясным только самому Иккю, когда он услышал
  именно эту песню, ему открылся смысл [коана] ``Ты заслужил
  шестьдесят ударов моей палки!'' 
\end{ver}

\begin{ver}
  До этого случая, Иккю звался Сюкен, но Касобыл настолько впечатлён
  случившимся, что /наградил/ своего ученика Буддистским именем Иккю,
  [что может быть примерно переведено как] ``Одна пауза'', что
  означает /передышка/\orig{repose} между жизнью и смертью,
  иллюзией/мороком/ и просветлением.
\end{ver}

