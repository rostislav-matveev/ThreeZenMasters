\str{18}
\begin{ver}
Несмотря на полученное прозрение, Иккю продолжал,
/страдать/бороться/испытывать затруднения/ с
/физическими/материальными/ и /духовными/умственными/ препятсвиями.
Он /стал регулярно/\orig{took to=взялся} медитировать /в
продолжение целых/\orig{troughout} ночей на озере Бива в лодке
одолженной у рыбака. В одну из ночей летом 1420го года, когда
двадцатишестилетний Иккю /дрейфовал/ в лодке по /неподвижному/тихому/ 
озеру, хрипло каркнула ворона. В это мгновение Иккю посетило
глубочайшее /осознание/инсайт/.

Когда Иккю рассказал о случившемся Касо, его учитель
/усмехнулся/фыркнул/: ``Может ты и архат, но всё же ты не мастер.''

[На что] Иккю беспечно ответил: ``Мне достаточно быть архатом. Кому
охота быть мастером?''

Касо остался доволен. ``Тогда ты действительно мастер'' -- сказал он. 
\end{ver}

\begin{ver}[1]
В 1420ом году в одну из летних ночей, 
\end{ver}

\begin{ver}
  Касо потребовал просветлённое стихотворение, как было тогда принято
  и Иккю написал:
\end{ver}

\begin{ver}
  \begin{verse}\it
    Десять лет я провёл в смятении\\
    Раздраженный\orig{seething} и злой; но теперь моё время пришло!\\
    Смеётся ворона, и из грязи /выходит/востаёт/появляется/ Архат,\\
    И в солнечном свете Чао-Ян, поёт нефритовая дева.
  \end{verse}
\end{ver}

\begin{ver}
  Последняя строка упоминает ещё одну легендарную покунутую красавицу
  древнего Китая, и, похоже, намекает на то, что просветлённый Иккю
  наконец-то примирился с судьбой своей матери.
\end{ver}

