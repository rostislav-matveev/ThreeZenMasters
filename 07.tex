\str{21}
\begin{ver}
После того как он /обрёл самостоятельность/, Иккю, по сообщениям,
встретился с, к тому времени /отрёкшимся/удалившимся/отставленным/
императором Го-Коматсу в 1427, что можно рассматривать как примирение
между отцом и незаконнорожденым сыном.
Го-Коматсу /испрашивал/\orig{solicited} и получил совет Иккю по
некоторым важным светским и религиозным вопросам, и после этого они
встречались /похоже/вероятно/\orig{apparently} регулярно.
Иккю был призван к /смертельному ложу/ Го-коматсу в 1433ем году и ему были
преподнесены несколько /картин/рисунков/ и каллиграфических работ из
имперской коллекции. Иккю, который ``не имел и собственной булавки'',
любовно хранил драгоценные свитки /всю свою жизнь/до конца своих дней/.
\end{ver}
