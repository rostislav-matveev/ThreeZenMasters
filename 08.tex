\str{21}
\begin{ver}
  Касо умер в 1428. Его прощальное стихотворение было
\end{ver}

\begin{ver}
\begin{verse}\it
  Капля воды /замерзает/замёрзла/ в воздухе\orig{in midair}\\
  Мои семьдесят семь лет\\
  Полностью использованы\\
  Весенний /вода/источник/ /пузырится/кипит/бьёт ключом/журчит/ /на
  огне/в пламени/  
\end{verse}
\end{ver}

\begin{ver}
Иккю присутствовал на похоронах своего учителя, после которых все
ученики Касо /разошлись, каждый своей дорогой/.
Сам Иккю никогда не осел, проводя жизнь /бродя/странствуя/скитаясь/
вокруг Киото, как он сам себя называл ``Сумасшедшее облако.'' 
\end{ver}

\begin{ver}
  \begin{verse}\it
    Сумасшедшее облако в открытом пространстве\\
    Бешено носимое [ветром], такой ???\orig{wild} как только бывает!\\
    Кто знает где это облако
    /остановится/осядет/накопится/стянется/\orig{gather}, где ветер
    /осядет/успокоится/?\\
    Солнце восходит из-зы восточного моря и освещает землю.
  \end{verse}
\comm{Нужен ориг подстрочник, wild имеет очень много значений}
\end{ver}

