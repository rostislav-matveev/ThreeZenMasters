\str{22}
\begin{ver}
  /В результате/В силу/ своей популярности,  в 1440ом году Иккю /был
  призван/ /стать/служить/ настоятелем /Ниой-ан/\comm{транслит!} --
  зависимым храмом Дайтоку--джи, но он быстро стал возмущён /фарсом/ и
  лицемерием окружавшим его имя и внезапно объявил о своей
  /отставке/уходе/ таким стихотворением:
\end{ver}
\begin{ver}
  \begin{verse}\it
    Десять дней в этом храме и мой ум кружится!\\
    Красная нить между ног тянется и тянется.\\
    В другой раз, когда будете меня искать,\\
    Посмотрите лучше в рыбной лавке, лотке с саке или в борделе
  \end{verse}
\end{ver}

\begin{ver}
  ``Красная нить /желания/страсти/похоти/'' ссылается на коан
  китайского мастера Сунь-юаня (---1202): 
\end{ver}

\begin{ver}
  \begin{quote}\it
    Чтобы познать Путь в совершенстве, есть один
    /момент/вопрос/особенность/ который необходимо
    /прояснить/\orig{penetrate} и нельзя избегать: красная нить
    желания между наших ног, которая не может
    /отрублена/перерезана/оторвана/.
    Лишь немногие готовы встретить эту проблему лицом к лицу\comm{см
      ориг}, так как её не так просто разрешить.
    Но за эту проблему надо браться прямо, без колебания, не
    отступаясь. Иначе откуда же взяться освобождению?
    \comm{Внимательно см ориг}
  \end{quote}
\end{ver}

\begin{ver}
  В краткой форме этот коан звучит так:``Почему даже наиболее
  просветлённый не может отрезать красную нить желания?''
\end{ver}
