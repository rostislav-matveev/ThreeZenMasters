\str{25}
\begin{ver}
Лин-цзы был /груб/резок/ и агресивен со своими
/нерешительными/колеблющимися/ учениками, избивая их кулаками, и
оглушая /громоподобным/протыкающим уши/ криком. Он был
страшен\orig{truly a terror}, как свидельствует надпися Иккю на
портрете  грозного мастера. 
\end{ver}

\begin{ver}
  \begin{verse}\it
    КАТСУ, КАТСУ, КАТСУ, КАТСУ!\\
    В зависимости от обстоятельств он убивает или оживляет!\\
    /Ужасный/злой/ бес с проникающими глазами,\\
    Которые видят так же хорошо, как солнце и луна.
  \end{verse}
\end{ver}

\begin{ver}[1]
  \begin{verse}\it
    БАМ, БАМ, БАМ, БАМ!\\
    В зависимости от обстоятельств он убивает или оживляет!\\
    Страшный бес с глазами как иглы,\\
    Их взгляд везде, как свет от солнца и луны.
  \end{verse}
\end{ver}

\begin{ver}[1]
  \begin{verse}\it
    БАМ, БАМ, БАМ, БАМ!\\
    Он дарит жизнь и убивает -- как повезёт!\\
    Ужасный демон, глаза как иглы,\\
    Их взгляд везде, как свет от солнца и луны.
  \end{verse}
\end{ver}

\begin{ver}
  Иккю считал себя одним из настоящих наследников Лин-цзы и назвал
  своё любимое место уединения -- /хижину/скит?/ Катсуро-ан --
  ``Хижина Слепого Осла'', согласно предсказанию Лин-цзы что его
  учение будет передаваться слепыми ослами -- упрямыми,
  бескомпромисными последователями Дзен, кто не будет ослеплён славой
  или богатством. С этой позиции, Иккю позволял себе
  /бранить/накасывать/ дилетантов от Дзен.
\end{ver}

