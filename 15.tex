\str{26}
\begin{ver}
Иккю бы даже ближе с Шу-тянем, китайским учителем Даиё, в свою
очередь, учителем Дайто, основателя линии
Дайтоку-джи. /Бродяга/странник/скиталец/ и поэт, Шу-тянь настаивал на
чистоте Дзена: ``Закон Будды в том, чтобы поступать правильно и
надлежаще, а не в постройке чрезмерных\orig{lavish} зданий и не в
прихотливых\orig{fancy} титулах.''
/Неистово/отчаянно/яростно/ независимый и неподчиняющийся ни светским,
ни религиозным властям, Шу-тянь был однажды посажен в тюрьму за
неподчинение государственному декрету, которому он считал
несправедливым.
Его прощальное стихотворение было просто:
\end{ver}


\begin{ver}
  \begin{verse}\it
    Восемдесят пять лет\\
    /Не осведомлённый о/ничего не зная о/невежественный в/несведущий
    в/ Буддизме,\\{}
    [Просто] иду /неуклонно/постоянно/постепенно/,\\
    Не оставляя следа в пустоте
  \end{verse}
\end{ver}

\begin{ver}
  {\em Записки о Шу-тяне (яп. Кидо року)} были любимым чтением Иккю.
  Ниже приводятся ``Три Ключевых Изречения'' с комментариями Иккю: 
\end{ver}

\begin{ver}
  \begin{quote}
    \begin{enumerate}
    \item 
      ``Если не обладаешь ясным взором, то как сделать полотняные
      штаны из воздуха?''
      %\smallskip\par
      \begin{verse}\it
        Рисунок рисовой лепёшки -- жестокая шутка -- не удовлетворяет
        голода.\\
        Рождён с глазами, а живёшь слепым.\\
        В холодной зале, представь себе одежды,\\
        И мантия бессмертных /проступит/появится/ /в/из/ темноте.
      \end{verse}

    \item
      ``Окруженный /линией/чертой/ на песке, как выйти не переступив?''
      \begin{verse}\it
        Я никогда не устаю от /великих/ радостей весны,\\
        Но все остальные, боятся испить эту чашу.\\
        Рай наступил, а преисподняя исчезла.\\
        Я провожу время среди падающих лепестков и летящего пуха.
      \end{verse}
      
    \item
      ``Можешь сосчитать песчинки на берегу, но как сможешь устоять на
      кончике иглы?''
      \begin{verse}\it
        Перекопай землю, пересчитай песчинки, это важное дело;\\
        Обладая сверхъестественными способностями, можно устоять на
        игле.\\
        А я\ldots Я просто /неотёсанный/нескладный/грубый/простой/
        бесталанный монах,\\
        И всё же, я преемник Шу-тяня в Японии. 
      \end{verse}
    \end{enumerate}
  \end{quote}
\end{ver}

\begin{ver}
  Вскоре Иккю начал подписывать свои работы, ``Иккю, воплощение
  Шу-таня в Японии в седьмом поколении.'' Он отрастил длинные волосы и
  стал носить нечесанную бороду, наподобие Шу-таня. Существует портрет
  ``Иккю в виде Шу-таня'' где оба лица объеденены в одно. Иккю сочинил
  следующее стихотворение в честь Шу-таня:
\end{ver}

\begin{ver}
  \begin{verse}\it
    Преподобный Шу-тань пренебрёг миром\\
    Отбросил свои робы как рваный сандаль\\
    Его не заботила ``Правильная Передача'' Лин-цзы.\\
    В восхищении небо сверкает и разражается песней
  \end{verse}
\end{ver}


