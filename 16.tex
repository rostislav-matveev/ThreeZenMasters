\str{27}
\begin{ver}
После того как Иккю /в гневе/с отвращением/ ушёл из Ниои-ан в 1440
году, он /обнародовал/распростронял/провозглашал/ Дзен стиля
Лин-цзы-Шу-тянь-Иккю в Киото и окресностях. В один из новогодних 
дней, которые совпадали с его днём рождения, Иккю прикрепил череп к
бамбуковой палке и маршировал по городу /крича/скандируя/:
``/Остерегайтесь/Будте осторожны/!Остерегайтесь!''. Когда празднующие
горожане стали /сетовать/пенять/, что он портит праздник, Иккю
отвечал: ``Напоминание о смерти не портит праздника,'' добавляя: ``Я
тоже праздную.'' И он прочел вот это стихотворение:
\end{ver}

\begin{ver}
  \begin{verse}\it
    Из всех вешей\\
    Нет ничего\\
    Более /удачного/счастливого/\orig{felicitous}\\
    чем этот /изношенный/потёртый/побитый ветрами/\\
    старый череп
  \end{verse}
\end{ver}

\begin{ver}
``Только если вы хорошо понимаете и принимаете это, вы можете
по-настояшему праздновать Новый Год,'' -- /наставлял/убеждал/ Иккю.
\end{ver}

\begin{ver}
Когда в один из годов, Иккю проживал в Сакае, он всегда носил с собой
деревянный меч.

-- Почему ты так поступаешь? -- спрашивали его. -- Меч это орудие
убийства и врядли подходит монаху.

-- Пока этот меч в ножнах, -- объяснял Иккю, -- он выглядит как настоящий
и производит впечатление. Но если его вынуть, то он просто деревянная
палка и не более чем посмешище -- вот как выглядит Буддизм в наши дни
-- роскошный\orig{splendid} снаружи и
/гнилой/\orig{transparent=прозрачный} внутри.
\end{ver}
