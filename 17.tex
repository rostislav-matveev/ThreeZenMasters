\str{28}
\begin{ver}
Однажды богатый купец устроил вегитарианский пир, на который были
приглашены ряд настоятелей и именитых священнослужителей.
Когда пришёл Иккю в потёртой робе и рваных сандалиях, его приняли за
обычного попрошайку дали медяк и выпроводили.

На следующий раз Иккю явился в прихотливых одеяниях, был принят и
положил одежды перед подносом с едой.

-- ``Что это ты делаешь?'' -- удивился хозяин.

-- ``Еда принадлежит этим робам, а не мне.'' -- сказал Иккю, удаляясь.
\end{ver}

\begin{ver}
  Каждый год храмы на горе Хиеи, выносили свитки с сутрами на
  проветривание, и множество пилигримов /посещало/ гору в этот день,
  чтобы получить заслуги. Однажды и Иккю присутствовал и когда он заснул под
  деревом недалеко от главного храма, священослужитель стал его гнать:
  ``Сегодня мы проветриваем священные тексты. Тут тебе не место
  дремать, ты -- кощунственный негодяй!'' Иккю встрепенулся и
  /возразил/: ``Я тут проветриваю Буддизм во плоти, а не напечатанный
  на бумаге. Так что оставь меня в покое!''  
\end{ver}

\begin{ver}
  Кроме вина и /женщин/женского общества/ Иккю очень любил
  /морепродукты/, особенно осьминогов \comm{Может кальмаров?
  Осьминогов редко едят.}, и сочинил такое стихотворение в
  честь своего любимого лакомства:
\end{ver}

\begin{ver}
  \begin{verse}\it
  Много рук, как у Богини Каннон;\\
  Принесён мне в жертву, /украшен/гарнирован/приправлен/ лимоном, так
  /чту/благоговею/почитаю/его!\\
  Вкус моря, просто божественно!\\
  Извини меня Будда, этот завет я не могу соблюдать.
  \end{verse}
\end{ver}
\comm{Всё неточно. См. ориг.}
