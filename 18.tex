\str{30}
\begin{ver}
У одного Киотского аптекаря было отличное средство от больного горла,
но он сохранял рецепт в большой тайне, и брал за лекарство огромные
деньги.
Иккю не нравилась такая скаредность\comm{не точно}, и он пошёл
навестить его.
``Тысячи могли бы получить пользу от твоего лекарства,'' -- увещевал
его Иккю. -- ``Нехорошо думать лишь о себе.'' Но аптекарь всё равно
отказывался поделиться рецептом, но когда Иккю настоял, то аптекарь
открыл Иккю формулу, с условием, что тот будет ей пользоваться только
для себя и никому не откроет секрет. Иккю согласился. Вернувшись
домой, однако, Иккю тут же выставил знак гласящий:
\end{ver}

\begin{ver}
  \begin{quote}
    Вниманию всех страдающих горлом!\\
    Приготовте лекарство по приведённому ниже рецепту и вскорости вы
    почувствуете облегчение.
  \end{quote}
\end{ver}

\begin{ver}
Услышав об этом, аптекарь страшно разозлился и /стал
пенять/\orig{confronted} Иккю: ``Ты же обещал никому не говорить,'' --
кричал он. ``Я не произнёс и слова,'' -- возражал Иккю. -- ``Всё что я
сделал, это повесил объявление. И я оказал тебе большую услугу -- если
бы ты и дальше держал рецепт в секрете, то оказался бы в аду. Я спас
тебя от этой участи.'' 
\end{ver}

\begin{ver}
  Времена были хаотичны, но Иккю не искал бездействия в
  убежище\comm{ориг}.
  Он взаимодействовал с обществом на всех уровнях и практиковал
  энергичный\orig{fierce}, вовлеченный Дзен.\comm{orig}
  Крестьянство было обескровлено черезмерными поборами и продажными
  чиновниками, и Иккю часто заступался за них /посылая/\orig{issuing}
  поэтические протесты местным /владыкам/феодалам/\orig{lords}. 
\end{ver}


\begin{ver}
  \begin{verse}\it
    Снова и снова;\\
    Берут и берут\\
    Из этой деревни;\\
    Умори их голодом\\
    И как {\em ты} будешь жить?
  \end{verse}
\end{ver}

\begin{ver}[1]
  \begin{verse}\it
    Снова и снова;\\
    Берёшь и берёшь\\
    Из этой деревни;\\
    Умори их голодом\\
    И как {\em сам} будешь жить?
  \end{verse}
\end{ver}


