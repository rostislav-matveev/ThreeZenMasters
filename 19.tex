\str{31}
\begin{ver}
Иногда Иккю /представлялся/ /Робином Гудом/Робин-Гудом/. Когда умер
один богатый купец, его семья хотела устроить ему
/сложные/\orig{elaborate} и дорогостоящие похороны, но Иккю
/стянул/украл/\orig{pilfered} деньги и раздал /неимущим/беднякам/.
Когда семья стала протестовать Иккю /стал
выговаривать/\orig{reprimanded} им: ``Одной монеты достаточно, чтобы
заплатить паромщику через Стикс. Деньги надо тратить на живых, а не мёртвых!''
\end{ver}

\begin{ver}
  Иккю обратился с таким стихотворением к властям:
  \begin{verse}\it
    \centerline{``Простить Все Долги''}
    Грабители не /разоряют/расхищают/ домов бедняков;\\
    Личное богатство не  приносит пользы всем\orig{rest of the
      nation}\\
    Источником бед являются сокровища накопленые
    одиночками\orig{few},\\
    Которые потеряли душу за десят тысяч монет.
  \end{verse}
\end{ver}

\begin{ver}
  Иккю и сам был хорошо знаком с нищетой, которая [тогда] была широко
  распространена. 
\end{ver}

\begin{ver}
  \begin{verse}\it
    \centerline{``Стихотворение в Обмен на Еду''}
    Опять я брожу по Хигашияма голодный.\\
    Когда голодаешь, миска риса дороже тысячи золотых монет.\\
    В древности один достойный человек /поменял/променял/ свою
    мудрость на несколько плодов личи\\
    И всё же я не могу удержаться чтобы воспеть ветер и луну.\\
  \end{verse}
\end{ver}

\begin{ver}
  \begin{verse}\it
    \centerline{``В Благодарность за Подарок Соевый Соус''}
    Ничем не связанный, свободный, вот уж тридцать лет,\\
    Сумасшедшее Облако практикует свой собственный вид Дзен.\\
    Приправой из тысячи трав дль меня являются:\\
    Жидкая похлёбка, да чай из прутьев это часть Передачи Учения\\
    \comm{см. ориг. Не очень точно/понятно}
  \end{verse}
\end{ver}

