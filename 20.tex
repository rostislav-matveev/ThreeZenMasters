\str{32}
\begin{ver}
Вместо золота и серебра Иккю побуждал своих последивателей собирать
``джентельменское богатство'':
\end{ver}

\begin{ver}
  \begin{verse}\it
  Сокровище поэта состоит из слов и фраз;\\
  Дни и месяцы ученого надушены книгами.\\
  Цветы вишни за окном:
  Непревзойденное удовольствие.\\
  {}[Хоть] живот свело от /холода/голода/, но всё равно очарован снегом
  и луной и изморозью на восходе 
  \end{verse}
\end{ver}

\begin{ver}
  Забота Иккю о чувствующих существах распростронялась и на животных,
  и иногда он проводил погребальные обряды для домашних
  животных. Когда умер его собственный воробей, Иккю написал ткой
  свиток в его память:
\end{ver}

\begin{ver}
  \begin{quote}
    \centerline{``Благородный обитатель леса''}
    Я вырастил маленького воробья и глубоко любил его. Однажды он
    неожиданно умер и убитый горем от этой потери я решил провести
    погребальный обряд для моего маленького друга, как если бы он был
    человеком. Сначала я называл его ``Воробей Ученик'', но после
    смерти я переименовал его в ``Воробей Будда''. В конце концов, я
    представил ему имя ``Благородный Обитатель Леса'' и написал такое
    мемориальное стихотворение:
    \begin{verse}\it
      Шестнадцатифутовое тело Будды в пурпуре и золоте\\
      Лежит между парой деревьев в Нирване.\\
      Теперь освобождённый от фальши, за жизнью и смертью\\
      Но присутсвующий в тысячах гор, десяти тысячах деревьев и сотнях источников.
    \end{verse}
  \end{quote}
\end{ver}
