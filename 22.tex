\str{35}
\begin{ver}
Хотя Иккю никогда не /искал/хотел/ учеников и в действительности
/отговариавал/препятствовал/\orig{discourage} /приходящих/ищущих его/,
он всю-таки был постоянно окружён
/учениками/сторонниками/последователями/\orig{disciples}
преуспевающими /в/на/ аскетизме и /цчистой практике/, которую они
быучили у него. Как выразился сам Иккю, ``Упорное старание это
суть\orig{essence} Будд и Патриархов; Буддами становятся, а не
рождаются''

Правда некоторые из них понимали Иккю не совсем
правильно\orig{misunderstood}. Один из учеников Иккю,
/копируя/повторяя/ /дикие/безумные/неистовые/ поступки Иккю, спал на
алтаре среди образов Будд, и использовал страницы из сутр в качестве
туалетной бумаги. Иккю призвал его и спросил:

-- Как ты думаешь, ты Будда?

-- Да, -- отвечал тот. -- Мы все Будды -- Вы сами нам это говорили.

-- Если ты Будда, зачем использовать грязную бумагу из свитков в
качестве туалетной бумаги? Разве Будда не заслуживает чего-то лучшего,
как, например, чистой белой бумаги? -- спросил его Иккю и потребовал: -- Прочти-ка
мне твои строки /просветления/прозрения/.

Ученик /отреагировал/отозвался/откликнулся/:
\end{ver}

\begin{ver}
  \begin{verse}
    Сижу в задзен\\
    /У/На/ мосту\\
    У четвёртой и пятой авеню\\
    Все люди, проходящие мимо\\
    Для меня просто как деревья глубоко в лесу. 
  \end{verse}
\end{ver}

\begin{ver}
  -- /Плохо/Никуда не годится/! -- закричал Иккю и прочитал свою
/отредактированную/исправленый/ вариант
\end{ver}

\begin{ver}
  \begin{verse}
    Сижу в дзадзен\\
    /У/На/ мосту\\
    У четвёртой и пятой авеню\\
    Все люди, проходящие мимо\\
    Для меня такие, какие есть! 
  \end{verse}
\end{ver}

\begin{ver}
  После этого, ученик уже не копировал поведение Иккю.
\end{ver}

\begin{ver}
  В 1444 возникла серьёзная угроза независимости Дайтоку-дзи, и Иккю
  вместе с Ёсо приложил усилия, чтобы воспрепятствовать назначению
  Ниппо (1368---1448) настоятелем Дайтоку-дзи. Ниппо принадлежал
  враждебной групировке из Миёсин-дзи, покровителствуемой сёгунами,
  тогда как Дайтоку-дзи находилсяа под опёкой императора.  Иккю и Ёсо
  даже /репетировали/ возможность /физически/ (Иккю /блокирует ворота)
  и психологически (их острые вопросы остановят Ниппо) не допустить
  Ниппо вступить в должность настоятеля монастыря. Их попытки остались
  бесплодными так как Ниппо был поддержан влиятельным\orig{powerful}
  кланом Хосокава, приверженцами сёгуна, и узурпатор из Миёшин-дзи был
  всё-таки назначен настоятелем. Монахи из Дайтоку-дзи продолжали
  протестовать против Ниппо и, когда один покончил жизнь
  самоубийством, несколько было сделано козлами отпущения и заключено
  в тюрьму. Это переполнило чашу Иккю и он удалился на гору
  Юзуриха(?), поклявшись голодать до смерти в качестве протеста:
\end{ver}

\begin{ver}
  \begin{verse}\it
    Мне стыдно ставаться среди живых\\
    Я практиковал Дзен и учил Путь так долго, и всё же вся эта грязь.\\
    Истинный Закон Будды был смятён и уничтожен\\
    Заменен властителями демонов высотой в сто футов
  \end{verse}
\end{ver}

\begin{ver}
  Император Го-Ханазоно (/правил/\orig{r.} 1429---1464)
  /послал/отправил/ /гонца/посыльного/ Иккю с такой /мольбой/\orig{plea}:
\end{ver}

\begin{ver}
  \begin{quote}
    Если /глубокочтимый/уважаемый/ монах будет продолжать голодовку,
    то Пути Будды и Пути Императора исчезнут\orig{perish}. Как можете
    Вы покинуть нас? Как можете Вы оставить /нацию/народ/государство/\orig{nation}?
  \end{quote}
\end{ver}

\begin{ver}
  Возможно потому, что ему удалось выиграть очко, привлекши внимание
  императора (и кроме того положение дел в Дайтоку-дзи несколько
  улучшилось), Иккю /был уговорён/\orig{was persuaded} прекратить голодовку. 
\end{ver}
