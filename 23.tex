\str{37}
\begin{ver}
Около 1450 года, временное перемирие между Ёсо и Иккю закончилось и
старая вражда возобновилась. Несколько учеников престарелого Ёсо
отреклись от него и присоединились с сообществу учеников Иккю, что
добавило масла в огонь. В 1453 году, /чудовищный/гибельный/ пожар
разрушил/уничтожил/ Дайтоку-джи почти полностью, и когда один из
учеников Ёсо был назначен настоятелем для надзора за востановлением
[храма], /гнев/ярость/возмущение/ Иккю /стала нарастать/появилась/.
На следующий год, Иккю и Ёсо встретились лицом к лицу, где Ёсо обвинил
Иккю в том, что он бросает ``воду с говном'' в лицо их Учителя Касо,
своим диким поведением и постоянной критикой законных его наследников.
Иккю возражал, что последователи Ёсо даже не упоминают голода и
трудностей, которые /испытывали/ Дайто и другие патриархи Даитоку-джи
в поике истинного Дзен, а вместо этого живут за счёт влияния на
аристократию и сёгунов и грандиозных зданий воздвигнутых в их имя.
\end{ver}

\begin{ver}
  \begin{verse}\it
    [Иккю - js] говорит о непревзойдённом блеске Дайто\\
    Но грохот /королевских/ карет заглушает его\\
    Никто не слушает истории о Патриархов\\
    Долгих годах голода и бездомности под Мостом Пятой Авеню.
  \end{verse}
\end{ver}

\begin{ver}
  После этой [взаимно] оскорбительной /встречи/столкновения/ ни один
  из них не разговаривал с другим, хотя Иккю продолжал свои нападения
  в /язвительных/саркастических/ стихах и публичных заявлениях. Иккю
  даже, /из всех людей/ обвинял Ёсо в общении с женщинами и радовался,
  когда Ёсо умер в 1458 году, от, по утверждениям Иккю, проказы,
  справедливого наказания еретикам и злодеям.\comm{см. ориг} В чувстве
  сильнейшего отвращения от склоки с Ёсо и его преемниками, в 1461ом
  году Иккю временно оставил Дзен и на несколько месяцев присоеденился
  к секте Чистой Земли. 
\end{ver}

\begin{ver}
  \begin{verse}\it
    Сумасшедшее Облако это /бес/демон/ /в линии/среди последователей/
    Дайто\\
    Он ненавидит эти /чёртовы/адские/ /склоки/ссоры/\\
    К чему старые [добрые] коаны и устаревшие /прецеденты/методы/\\
    Нечего жаловаться. Я буду /полагаться/надеяться/ на мои
    /внутренние/собственные/ сокровища\\
  \end{verse}
\end{ver}

\begin{ver}
  В чём же причины непримиримой неприязни Иккю к Ёсо? Они были ярыми
  соперниками, будучи вместе учениками Касо, и Иккю несомненно
  чуствовал, вполне справедливо, что он был более талантлив, чем более
  старший Ёсо. Тем не менее это именно Ёсо был признан главным
  учеником Касо и был дважды назначен настоятелем Дайтоку-дзи. Также
  исключительный титул ``Мудрый Дзен Мастер Великого
  Света''\orig{``The Wise One, Great Light Zen Master''} был присвоем
  ему императором. (Иккю переставил иероглифы так, что получилось:
  ``Грязный Дзен Мастер Переполненый Страстями''\orig{``The Filthy
    One, Burning with Lust Zen Master''}.) Другим источником
  разногласий была разница их характеров, которые были противоположны:
  Иккю был /подвижным/живым/, бескомпромисным, и радицальным; Ёсо был
  примирительным, осторожным и /консервативным/традиционным/. С точки
  зрения Иккю, Ёсо, несмотря на все его мирские титулы, был не более
  чем религиозной\orig{ecclesiastical} куклой, лишенной настоящего
  вдохновения и понимания [сути вещей], олицетворяющем всё то в
  области дзен, к чему Иккю питал такое отвращение. И тем не менее,
  враждебность Иккю производит впечатление черезмерной и демонстрирует
  изъян в его /личности/характере/.
\end{ver}

