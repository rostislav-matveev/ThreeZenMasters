\str{39}
\begin{ver}
Так как Иккю жил в один из самых /турбулентных/беспокойных/ периодов в истории
Японии -- непрекращающийся круговорот голода, чумы, рисовых
/бунтов/мятежей/восстаний/ и войн -- возможно его нетерпимость
/объяснима/понимаема/. В добавок к смерти и разрухи его окружавших,
Иккю постояна вступал в перепалки с духовными лицами, обвиняя их в
/искажении/извращении/ Буддизма ради богатства и славы.
В ресультате\orig{given} такого мрачных обстоятельств, в 1457ом году
шестидесятитрёхлетний пессимист почувствовал необходимость
сочинить \textit{``Скелеты'' (Гайкотсу)}, который стал его наиболее
известной работой, читаемой и изучаемой по сей день:
\end{ver}

\comm{Наверняка есть много переводов, но я попробую их пока не читать} 
\begin{ver}
\begin{quote}
  Эти тонкие линии индйских чернил открывают всю истину. 
  
  Ученики, настоятельно сидите в зазен, и вы /поймёте/осознаете/, что
  всё /рождённое/появившееся/ в этом мире в конечном счёте пусто,
  включая и вас самих и саму изначальную грань существования.
  /Всё/все вещи/ возникают  на самом деле из пустоты. Эта
  первоначальная бесформенность это [и есть] Будда, и другие похожие
  слова естество Будды, /Буддовость/\orig{Buddhahood}, ум Будды,
  Просветлённый, Патриарх, Бог -- всего лишь другие выражения для той
  же пустоты. /Поймите это неправильно/ и окажитесь в аду.
  \end{quote}
  \end{ver}

\begin{ver}[1]
\begin{quote}
  Эти тонкие черные, знаки содержат в себе всю истину.

  Медитируйте настоятельно и вы поймёте, что всё рождённое пусто,
  включая и вас самих и саму изначальную /рябь/пелену/
  существования. Всё возникает из пустоты. Эта первоначальная
  бесформенность и есть Будда, и все другие похожие слова --
  Будда-природа, состояние Будды, Будда-ум, Просветлённый, Патриарх,
  Бог -- всего лишь разные названия этой пустоты. Непонявший этого
  окажется в аду.
\end{quote}
\end{ver}

\begin{ver}
\begin{quote}
  Испытывая отвращение и страстно желая освободиться из водоворота
  рождений и смерти, я оставил свой дом и пустился в путь.
  В одну из ночей, когда я присматривал место для ночлега, я наткнулся
  на маленький одинокий храм. Он стоял у подножия горы, вдалеке от
  большой дороги, как будто затерявшись в безбрежной Долине
  Спокойствия.
  Вокруг было кладбище и вдруг жалко выглядящий скелет появился и
  сказал такие слова:
\end{quote}
\end{ver}

\begin{ver}
\begin{quote}
  \begin{verses}
    Меланхоличный осенний ветер,\\
    Дует через весь Мир;\\
    Волнуется трава в пампе,\\
    Пока мы дрейфуем в пустошь,\\
    Дрейфуем в море\\
    
    Что может быть сделано\\
    С /умом/сознанием/ человека\\
    Который должен быть /чист/спокоен/\\
    Но несмотря на то, что он [человек] облачён в монашескую робу,\\
    Он даёт жизни проходить мимо.
  \end{verses}
\end{quote}
\end{ver}

\begin{ver}
\begin{quote}
  Все вещи исчезают, возвращаясь к своему
  /началу/источнику/. Бодхидхарма сидел лицом к стене в глубокой
  медитации, и ни одна мысль возникшая у него в голове не имела
  реальной природы. То же самое относится к пятидесяти годам, [в
    течение которых] Будда провозглашал /Учение/Закон/. Ум не
  ограничен такими условностями.  
\end{quote}
\end{ver}

\begin{ver}
\begin{quote}
  Эти глубокие размышления беспокоили меня и я не мог заснуть. К
  рассвету я задремал и во сне оказался окружённым скелетами, ведущими
  себя так, как будто они ещё живы. Один из них подошёл ко мне и сказал:
\end{quote}
\end{ver}

\begin{ver}
\begin{quote}
  \begin{verses}
  Воспоминания\\
  Убегают и\\
  Их больше нет.\\
  Все они лишь пустые мечты\\
  Лишённые смысла.\\

  Нарушь реальность вещей\\
  /бормочи/рассуждай/ о\\
  ``Боге'' или ``Будде''\\
  И никогда не найдёшь\\
  Истинного Пути.\\
  
  Пока дышишь,\\
  Чувствуешь себя живым,\\
  И труп [валяющийся] в поле\\
  Кажется чем-то\\
  Не имеющим к тебе отношения.
  \end{verses}
\end{quote}
\end{ver}

\begin{ver}
  \begin{quote}
    Этот скелет понравился мне -- он отказался от мирского для поиска
    истины, и /оставил мелкое перейдя к глубокому/. Он всё видел ясно,
    так как оно есть. Я лежал, а ветер [в ветвях] сосен шептал мне и
    осенний лунный свет плясал на моём лице. 
  \end{quote}
\end{ver}

\str{41}
\begin{ver}
  \begin{quote}
    Что не является сном/миражом/иллюзией/? Кто [в конце концов] не
    превратится в скелет?
    Мы все являемся скелетами покрытыми кожей -- женщины и мужчины --
    и вожделеем друг друга. Когда же дыхание иссякает, кожа
    /разрывается/слезает/исчезает/\orig{raptures}, пол исчезает и нет
    более ни высокого, ни низкого. Под кожей, человек которого мы
    сейчас гладим и ласкаем, не более чем [просто] груда
    костей. Подумайте об этом -- высокие и низкие, молодые и старые,
    мужчины и женщины, все одинаковы. 
    Осознайте эту важнейшую вещь и сразу же поймёте смысл [фразы] ``не
    рожденный и не умираюший.''
  \end{quote}
\end{ver}

\begin{ver}
  \begin{quote}
    \begin{verses}
      Если куски камня\\
      Могут служить в качестве /памятников/\\
      Мёрвым,\\
      То лучшим надгробием\\
      Была бы обычная /ступка/\orig{tea-mortar}
    \end{verses}
  \end{quote}
\end{ver}

\begin{ver}
  \begin{quote}
    Люди и в самом деле ужасные существа.
  \end{quote}
\end{ver}

\begin{ver}
  \begin{quote}
    \begin{verses}
      Одинокая луна\\
      Яркая и чистая\\
      В безоблачном небе;\\
      И всё ж мы /бредём спотыкаясь/\orig{stumble}\\
      Во тьме /покрывающей [весь] мир/.
    \end{verses}
  \end{quote}
\end{ver}

\begin{ver}
  Посмотрите хорошенько -- остановить дыхание, /снять/\orig{peel off}
  кожу, и всё выглядит совершенно одинаково. Сколько бы вы ни жили,
  результат один и тот же [даже и для имепраторов -- js?]. Отбрось
  /понятие/идею/\orig{notion} ``Я существую.'' Доверься гонимым ветром
  облакам и не желай себе вечной жизни. 
\end{ver}

\begin{ver}
  \begin{verses}
    Этот мир\\
    не более чем\\
    Мимолетный/эфемерный/ сон.\\
    Зачем [тогда] тревожиться\\
    Из-за его /мимолетности/преходимости/?
  \end{verses}
\end{ver}

\str{42}
\begin{ver}
  \begin{quote}
    Продолжительность жизни предопределена и все мольбы богам о
    продлении бесполезны. Держите ум сконцентрированным на одном
    большом вопросе [жизни и смерти, -- js]. Жизнь заканчивается
    смертью, /так устроен мир/\orig{the way things are}. 
  \end{quote}
\end{ver}

\begin{ver}
  \begin{quote}
    \begin{verses}
      Превратности жизни,\\
      Хоть и болезненны\\
      Учат нас\\
      не /цепляться/держаться/\\
      за этот /уплывающий/изменчивый/блуждающий/ мир
      
      Почему люди\\
      /Ценят/\orig{lavish=расточать} это убранство\\
      на этой горе костей,\\
      Которое обречено исчезнуть,\\
      Не оставив следа?
      
      Изначальное тело\\
      Должно вернуться \\
      к своему источнику.\\
      Не ищите\\
      То, что не может быть найдено
      
      Никому неизвестны\\
      Ни /природа/сущность/ рождения\\
      Ни место, куда мы идём\orig{the true dwelling place}\\
      Мы возвращаемся к источнику\\
      И превращаемся в пыль.
      
      Много тропинок/путей/ идут\\
      От подножия горы,\\
      Но на вершине\\
      Мы любуемся\orig{gaze}\\
      Одной [и той же] яркой луной
      
      Если в конце нашего пути\\
      Нет конечного\\
      Места успокоения,\\
      То и нет необходимости бояться\\
      /Потерять[ся]/сбиться с Пути/
      
      Ни начала,\\
      Ни конца.\\
      Наш ум\\
      Рождён и умрёт:\\
      Пустота пустоты!
      
      Расслабься,\\
      И ум\\
      Несётся вскачь\orig{runs wild};\\
      Контролируй мир\\
      И сможешь его отбросить.\\
      
      Дождь, град, снег или лёд\\
      Все [такие] разные\\
      Но упав\\
      Превращаются в такую же воду\\
      Как в потоке в ущелье\\
      
      Способы объяснить\\
      Что такое Ум, различны,\\
      Но одна и та же /великая/\orig{hеаvenly} истина\\
      Просвечивает\\
      В каждом из них\\
      
      Покрой свои следы\orig{path}\\
      Осыпавшейся хвоёй\\
      Чтобы никто не смог\\
      Найти\\
      Твоё настоящее местопребывание\\
      
      Как /пусты/тщетны/суетны/бесплодны/тщеславны/\orig{vain}\\
      Бесконечные похороны\\
      В погребальных залах Горы Торибе.\\
      Неужто оплакивающие не понимают\\
      Что теперь будет их очередь?
     
      ``Жизнь преходяща!''\\
      Думаем мы при виде\\
      Дыма поднимающегося от горы Торибе,\\
      Но когда же поймём,\\
      Что и мы сами в той же лодке
      
      Всё тщетно!\\
      Этим утром\\
      Здоровый друг\\
      Этим вечером\\
      Клочок\orig{wisp} погребального дыма

      Какая жалость!\\
      Вечерний дым с горы Торибе\\
      Гонимый неистово\\
      Туда и сюда\\
      Ветром

      Сгорев\\
      Мы становимся пеплом\\
      Или землёй если похоронены\\
      Не одни ли наши грехи\\
      Остаются?

      Все грехи\\
      Совершённые\\
      В трёх мирах\\
      Исчезнут\\
      Вместе со мной
    \end{verses}
  \end{quote}
\end{ver}




\str{44}
\begin{ver}
  \begin{quote}
    Так устроен мир. Те, кто не
    /понял/осознал/усвоил/постиг/осмыслил/\orig{grasped} непостоянство
    мира, изумлены и повергнуты в ужас этими изменениями. Ныне лишь
    немногие ищут Буддисткую /правду/мудрость/ и монастыри почти
    пусты.
    
    Служители невежественны, и гнушаются зазен, считая его докучливым;
    они оставили медитацию, сосредоточившись на украшении своих
    храмов.
    Их зазен это мошенничество и они лишь притворяются монахами -- те
    робы, что они носят, однажды превратятся в тяжёлые ???\orig{coats
      of torture}.
  \end{quote}
\end{ver}

\begin{ver}
  \begin{quote}
    Во вселенной жизни и смерти убийство\orig{taking of life} ведёт в
    ад; жадность ведёт к перерождению в виде [вечно]-голодного
    /призрака/духа/; невежество -- к перерождению в виде животного;
    злоба превращает в демона. Следуй заповедям и дотигнешь
    перерождения в виде человека. Делай добро и [в следуюшей жизни]
    достигнешь уровня богов. Кроме этих шести миров\orig{realms} есть
    ещё четыре уровня Мудрых Будд, всего десять миров существования.
    И всё же одной просветлённой мысли достаточно, чтобы понять, что
    эти десять миров бесформенны и их ничего не разделяет и в них нет
    ничего такого, что стоило бы ненавидеть, бояться или
    /желать/страждать/. Тогда понятно, что существование это лишь
    /прозрачное/едва видное/\orig{wispy} облако в бескрайнем небе или
    пузыри на воде. Если не возникает мысли в Уме, то нет и никаких
    элементов [из которых состоит мир]. Ум и /окружающий
    мир/\orig{objects} это одно и тоже, и оба абсолютно пусты.
  \end{quote}
\end{ver}

\str{45}
\begin{ver}
  \begin{quote}
    Человеческое рождение как огонь: отец это /кремень/огниво/, мать
    -- камень, и /получающаяся в
    результате/высеченная/\orig{resultant} искра -- ребёнок.
    Пламя возгорается из основных элементов и горит пока есть топливо.
    Любовь между отцом и матерью рождает искру жизни. Но и родители не
    имеют начала и они угасают; всё возникает из пустоты, источника
    всех форм. Освободись от формы и вернись в изначальное состояние
    бытия. Из этого состояния бьёт жизнь, но не держись и за это.
  \end{quote}
\end{ver}

\begin{ver}
  \begin{quote}
    \begin{verses}
      Сломай \\
      Вишнёвое дерево/сакуру/\\
      И не будет цветов\\
      Но весенний бриз\\
      Приносит мириады лепестков!\\
      \ \\
      Безо всякой лестницы\orig{bridge}\\
      Облака поднимаются без усилий\\
      В небо;\\
      Незачем опираться\\
      На то, чему учил Будда Готама\\
    \end{verses}
  \end{quote}
\end{ver}
\str{46}
\begin{ver}
  \begin{quote}
    Будда Готама проповедовал Путь\orig{Law} пятьдесят лет и, когда его
    ученик Кашьяпа спросил его, в чём ключ к его учению, Будда сказал:
    ``С начала и до конца я не произнёс ни одного слова'' -- и протянул
    цветок. Кашьяпа улыбнулся и Будда отдал ему цветок, сказав такие
    слова: ``Ты владеешь /удивительным/чудесным/ умом Истинного
    Закона.''
    ``Что ты имеешь в виду?'' -- спросил Кашьяпа. ``Пятьдесят лет моей
    проповеди,'' -- сказал ему Будда, -- ``были /заманинаванием/ тебя,
    подобно тому как привлекают ребёнка /на руки/ обещянием награды.''
  \end{quote}
\end{ver}

\begin{ver}
  \begin{quote}
    Этот цветок Закона Буддизма не может быть описан ни в
    материальных\orig{physical}, ментальных, или словесных
    терминах. Он не [принадележит] не материальному, ни духовному. Он
    не является интеллектуальным знанием. Наш Закон это Цветок Одной
    Колесницы, несущей всех Будд Прошлого, Настоящего и Будущего. Она
    содержит двадцать шесть Индийских и шесть Китайских Патриархов;
    это изначальная основа сушествования, -- всё что [вообще]
    существует.
    Восемь чувств, четыре времени года, четыре основных элемента
    (земля, вода, огонь и /воздух/ветер/), всё возникает из пустоты, но
    немногие понимают это. Ветер это дыхание, огонь -- движение, вода
    -- кровь; когда тело сожжено или похоронено, оно становитсь
    землёй. Но элементы тоже не имеют начала и не постоянны.
  \end{quote}
\end{ver}

\begin{ver}
  \begin{quote}
    \begin{verses}
      В этом мире\\
      Всё без исключения\\
      Не имеет раальной природы\orig{is unreal}\\
      Сама Смерть\\
      Лишь иллюзия\\
    \end{verses}
  \end{quote}
\end{ver}


\str{47}
\begin{ver}
  \begin{quote}
    Наваждение в том, что кажется будто тело умирает, а душа остаётся
    -- это величайшая ошибка.
    Просветлённые говорят, что и душа и тело исчезают
    вместе. ``Будда'' это пустота и небеса и земль возвращаются к
    первооснове. Я отложил 80,000 книг и /даю/дал/ вам сущность
    [учения] в этой тоненькой книжке. Это принесёт вам великое блаженство.
  \end{quote}
\end{ver}

\begin{ver}
  \begin{quote}
    \begin{verses}
      Написать текст\\
      Чтобы оставить /после себя/потомкам/\\
      Это ещё один вид наваждения\\
      Когда я проснусь, я узнаю что\\
      Прочитать это будет некому.
    \end{verses}
  \end{quote}
\end{ver}

