\str{47}
\begin{ver}
Буддизм Иккю не всегда был таким тёмным\orig{dark} и мрачным; даже в
наихудшие моменты были периоды залитые светом и осенённые\orig{graced}
возвышенной красотой. Однажды он написал, ``Радость /среди/в условиях/
страдания это знак школы Иккю.''
\end{ver}

\begin{ver}
Ценить и наслаждаться красотой во всех её проявлениях, было
центральным в учении Иккю, и он /многими способами/\orig{in many ways}
является крестным отцом искуства Дзен.
\end{ver}

\begin{ver}
  За исключением {\em Скелетов} проза Иккю /дидактична/нравоучительна/
  и на удивление традиционна; его пылкие стихи, напротив, лучшие в
  своём жанре. Иккю хорошо знал, что прежние мастера Дзен осуждали
  литературу и поэзию, как адские /изобретения/средства/:  
\end{ver}

\begin{ver}
  \begin{verse}\it
  Нынче монахи /тяжело/сильно/упорно/ учатся чтобы\\
  /Блестнуть/ввернуть/ удачной фразой и прославиться как поэты.\\
  В лачуге Безимного Облака нет таланта, зато /много
  аромата/\orig{lots of flavor}\\
  Пока он готовит чашу риса в старом побитом котле.
  \end{verse}
\end{ver}

\begin{ver}
  И всё же одержимость Иккю поэзией /не утихала/не ослабевала/:
\end{ver}


\begin{ver}
  \begin{verse}\it
    Радость и печаль; любовь и ненависть; свет и тень; жар и холод;
    счастье и гнев; Я и другой.\\
    Наслаждение красотой поэзии может и ведёт в ад.\\
    Но посмотри, что мы находим разбросанным на дороге:\\
    Цветы сливы и лепески персиков!
  \end{verse}
\end{ver}

\begin{ver}
  Бесплодное рифмоплётство, умная игра слов и
  /книжная/словесная/\orig{literary} претенциозность душат Дзен, в то
  время как безумные стихи Иккю это манифестация Дзен. Он полностью
  раскрывает себя в своих неистовых,
  /беспрепятсвенных/неудержных/\orig{untrammeled} строках,
  /отчитываясь/\orig{recording} в восторге  и
  отчаянии,  в своей любви и ненависти, в своей человечности и в
  Буддаподобии, с бесстрашной честностью и прямотой, требуемыми
  настоящим Дзен.
\end{ver}
