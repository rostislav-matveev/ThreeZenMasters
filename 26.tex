\str{49}
\begin{ver}
Большинство поэм Иккю были написаны в Китайском стиле стихосложения
(канши), но он также слагал и Японские стихи (вака) и сильно повлиял
на Соги (1421---1502), который был центральной фигурой в истории
Японской поэзии, и который учил Дзен у Иккю и /впитал/воспринял/ стиль Иккю:
/утонченный/изысканный/изящный/рафинированный/\orig{refined}, но
/смелый/дерзкий/отчетливый/\orig{bold},
/прямой/непосредственный/\orig{direct} и
/пронзительный/проникающий/\orig{penetrating}. 
Иккю так же может рассматриваться как святой-покровитель хайку,
сатирических сенрю и сумасбродных киёка, которые развились позднее.
Сцепленный стих -- импровизация, где первая 5/7/5-слоговая часть
сочинялась одним поэтом  и дополнялась 7/7-слоговой сочиненной вторым,
постепенно развилась в форму хайку, то есть первую 5/7/5шслоговую
часть взятую саму по себе, как мгновенное\orig{shortest} и
милое\orig{sweetest} выражение буддисткой Истины, уверенный ``Дзенский
окрик'', тот, что всегда пропагондировался Иккю. Пример Иккю, его
жизнь и его /стихи/поэзия/ предоставили богатый /прецедент/набор примеров/ и 
вдохновление для /хлёсткой/едкой/ сатиры сенрю и
клоунады\orig{zaniness} киёко, ``безумных стихов''.
\end{ver}

\begin{ver}
  В культуре Дальнего Востока\orig{in the Far East} каллиграфия
  /почитается/считается/ наивысшим искусством, и стиль\orig{brushwork}
  Иккю [остается] непревзойдённым по /силе/интенсивности/ и /бешеной
  отдаче/\orig{wild abandon}. /Он/Его искуство/ изобилует неистовой
  энергией, и течет как ряд горных ручьёв, резонирующих и
  неостановимых\comm{см ориг}. Каллиграфия Иккю требует всего внимания
  зрителя, и можно получить настоящее понимания глубины Дзен Безумного
  Облака сталкиваясь с его /каллиграфическими работами/\orig{brushwork}.  
\end{ver}

\begin{ver}
  Иккю также и рисовал и /имел определяющее влияние/\orig{guiding
    hand} в развитии школы Сога. Его /строгие/суровые/ и простые
  рисунки тушью, не оставляют зрителя равнодушным\orig{give no
    quarter=не давать пощады}, хотя они и проще, чем его
  каллиграфические работы. Иккю был первым мастер использовавшим
  каллигарфию и рисунок как средства передачи учения Дзен. Следуя
  примеру Иккю, каллиграфия и рисунок -- /визуальная/наглядная/
  проповедь -- стали главным способом [так называемых] ``передачи
  через мастерство''\orig{skillful means} исползуемым всеми великими
  мастерами для вдохновления и поучения учеников Дзен.
\end{ver}
