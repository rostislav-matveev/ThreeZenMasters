\str{51}
\begin{ver}
В другой раз, когда продажи упали в продающей веера лавке, владельцами
которой была престарелая пара, и они /могли всё потерять/были на грани
полного разорения/, хозяева пришли к Иккю просить [совета и] помощи. Иккю
сказал: ``Усыновите меня на несколько дней.'' Владельцы лавки так и
поступили, и Иккю повесил объявление у входа: ``Иккю был усыновлён
/этой семьёй/в этом доме/, и отпразднует это тем, что завтра подпишет
каждый веер купленный в этой лавке.'' Новость распространилась подобно
лесному пожару и на следующий день лавка была /завалена/забита/\orig{swamped}
покупателями. В этот день было продано вееров больше чем за год и их
дело было спасено\comm{не по тексту немного}. 

Главный настоятель [храма/монастыря на] горе Хие попросил Иккю
[сделать какую-нибудь] каллиграфию для храма, но Иккю считал
настоятеля несколько помпезным, так что он дал ему обрывок бумаги с
месивом неразборчивых каракулей. Когда же настоятель вежливо попросил
что-нибудь более величественное и белее разборчивое, Иккю потребовал
собрать всю бумагу, которую тот сможет собрать, а также самую большую
кисть. Иккю склеил все полученные листы в один гигантский свиток,
тянущийся от одного склона горы до другого. После этого /обмакнув/окунув/ кисть
в ведро с тушью он понесся вдоль одного края свитка, оставляя за собой
длинную черту. Окунув кисть в тушь ещё раз, он вернулся вдоль другого
края.

-- Что /это/тут написано/? -- спросил настоятель
/смущённо/застенчиво/\orig{sheepishly}.

-- Как же! Это [знак] {\em ши}, разумеется, -- отвечал Иккю со
смешком. ({\em ши} это самый простой знак в Японской слоговой азбуке
кана)
\comm{не точно по тексту}
\end{ver}

\begin{ver}
  Иккю оказал глубокое и устойчивое влияние на садо, искуство чайной
  церемонии. По его мненинию, приготовление и употребление чая должно
  служить как Дзен в действии; чайная церемония это нечто большее, чем
  приятное времяпрепровождение или возможность похвастаться
  привезённой из Китая старой и дорогостоящей керамикой. Употребление
  чая должно основываться на Дзенской эстетике: фурю (глубокое
  осознание основы), ваби (/незатейливая/строгая/суровая/ красота,
  /понимание ценности/\orig{appreciation} естественного и
  неприукрашенного), саби (/понимание ценности/\orig{appreciation}
  старого, выдержанного, /единственного в своём роде/\orig{solitary},
  тихого), и шибуи (вкус к /???/\orig{astringent=вяжущий вкус у
    зелёного чая} и /???/\orig{understated=недосказанный, не ярко
    выраженный}). Иккю высказывал что чайная церемония должна
  уравновешивать\orig{harmonize} лучшее из Китайского и Японского
  /вкуса/стиля/. 
\end{ver}

\begin{ver}
  Будучи первым, кто стал заказывать керамические [изделия] годные для
  чайной церемонии /у местных гончаров/\orig{from local kilns} Иккю
  стимулировал развитие Японского /гончарного
  искуства/\orig{pottery}. Иккю был учителем Мурато Шукё
  (1422---1502), первого официального мастера чайной церемонии в
  Японии, и свиток подаренный Иккю Шукё с каллиграфией Китайского
  мастера Юан-ву (Енго, 1063---1135) был тот самый, выставляемый на
  первых церемониях Шукё. Это /ознаменовало/положило начало/ традиции
  вешать /работу/каллиграфию/ [какого-нибудь] мастера Дзен в алькове
  во время церемонии. (Конкретно этот свиток до сих пор ценится как
  прародитель Дзенского искуства чайной церемонии.) Шукё равно как и
  монах Ётей (возможно\orig{thought to have been} сын Иккю) передали
  эти идеалы Такено Ёё, (1502---1555), кто в свою очередь, передал их
  патриарху современной японской школы чайной церемонии Сен-но-Рикю
  (1522---1591). 
\end{ver}

