\str{52}
\begin{ver}
Прославленный актер театра Но Конпару Зенчику (1405---1468) учил
Дзен [под руководством] Иккю, чтобы усовершенствовать свой
дух\orig{mind} и своё драматическое искуство. С тех пор Дзен связывают
с театром Но: актёр-мастер должен ухватить\orig{capture}
/идеальное/абсолютное/совершенное/ неподвижность /посреди/внутри/
движения, /настроить/согласовать/привести в соответствие/ свой ум и
тело с хором и аудиторией, так чтобы они стали одним. Две драмы Но,
``Егучи'' и ``Ямауба'', также  -- без окончательного на то
/обоснования/доказательства/ -- приписывают Иккю.   
\end{ver}

\begin{ver}
  Иккю была знакома изысканная красота любви. ``С мая по
  декабрь''-роман между молодой слепой менестрелью Леди Мори и старым
  настоятелем один из самых прославленных в Японии. Когда Иккю
  встретил её впервые он написал:
\end{ver}

\begin{ver}
  \begin{verse}\it
    Самая красивая и самая /настоящая/истинная/подлинная/\orig{truest}
    из всех женщин;\\
    Её песни это свежая, чистая мелодия любви.\\
    Голос и нежная\orig{sweet} улыбка, что разрывает моё сердце:\\
    Я в весеннем лесу кребовых яблонь
  \end{verse}
\end{ver}

\begin{ver}
  Захваченные чувством\orig{hopelessly smitten} эти двое стали неразлучны.
\end{ver}


\begin{ver}
  \begin{verse}\it
    Каждую ночь Слепая Мори сопровождает меня /с/в/ песне.\\
    Под покрывалом два голубя\orig{mandarin ducks} шепчут друг другу\\
    Мы обещаем остаться вместе навсегда\\
    И\orig{but} сейчас этот старик наслажается вечной весной
  \end{verse}
\end{ver}


\begin{ver}
  \begin{verse}\it
    У реки и у моря и в горах\\
    Человек Пути /избегает/гнушается/ славы и богатства\\
    Ночь за ночью мы два голубка\orig{lovebirds} льнём друг к другу на
    /платформе/\comm{жуть} для медитации\\
    Увлеченные флиртом, интимной болтовнёй\orig{intimate talk} и
    неземным блаженством.
  \end{verse}
\end{ver}

\begin{ver}
  Иккю, очевидно, имел от Мори ребёнка он посвятил своей дочери
  следующее стихотворение:
\end{ver}

\begin{ver}
  Даже среди красавиц она бесценная жемчужина\\
  Маленькая принцесса в этом /жалком/печальном/мрачном/ мире.\\
  Она неизбежный результат настоящей любви,\\
  И [никакой] Дзен мастер с ней не сравнится.
\end{ver}

\begin{ver}
  В его шестидесятые и семидесятые годы связь Иккю с Дзенским
  искусством принесла великолепные плоды, но тем временем мрачное и
  отчаянное\orig{desperate} положение дел в Киото ещё ухудшилось. Во
  время бессмысленной десятилетней войны Онин (1467---1477), Киото был
  полностью разрушен и покинут живыми\orig{anandoned to the dead}.
  И Катсуро-ан и Дайтоку-дзи сгорели до тла в первый же год конфликта
  и Иккю искал пристанища сначала в Сюон-ан, основанном им самим
  маленьком храме, а потом и в других близлежащих местах. В последние
  два десятилетия своей жизни Иккю был часто болен, мучаемый частыми
  приступами диарреи.
\end{ver}
