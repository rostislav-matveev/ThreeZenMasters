\str{57}
\begin{ver}
Настоящее прощальное стихотворение Иккю, написанное за несколько часов
до того, как в двадцать первый день, одиннадцатого месяца, 1481го
года, Иккю умер сидя в позе лотоса, такое:
\end{ver}

\begin{ver}
  \begin{verse}\it
    В этом гигантском космосе\\
    Кто понял мой Дзен?\\
    Даже если бы сам Хсю-тань вернулся\\
    Он не стоил бы и пол-монеты
  \end{verse}
\end{ver}

\begin{ver}[1]
  \begin{verse}\it
    Во всей безграничной вселенной\\
    Кто исчерпает мой Дзен?\\
    Сам Хсю-тань вернувшись,\\
    Не стоил бы и гроша.
  \end{verse}
\end{ver}

\begin{ver}
  Бокусай, бывший долгое время учеником Иккю и ставший впоследствии его
первым биографом, /восхвалял/\orig{eulogized} своего учителя так:
``Иккю не делал различия между низким и высоким в обществе, и получал
удовольствие от общения с ремеслиниками, купцами и детьми. Детвора
бегала за ним и птицы клевали корм из его рук. Всё что он получал, он
отдавал другим. Он был строг и требователен, но ко всем относился
/одинаково, без предубеждения/\orig{without favoritism}. Иккю смеялся
искренне, когда был счастлив и /кричал могуче/\orig{shouted mightily},
когда был сердит.''
\end{ver}

