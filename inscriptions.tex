\subsubsection{Страница 19}
\begin{ver}
Каллиграфия [кисти] Иккю, ``Иккю'' (``одна пауза'') (Музей искуств Фуджита,
Япония)  
\end{ver}

\begin{ver}
/Один/отдельный/ /Штрих/мазок кисти/ обозначающий /``один''/``одна''/
также служит фаллическим символом, [в соответстсвии с тем, что] секс
это существенный элемент Дзен, [стиля] Иккю. [В] Стихотворение над
именем [Иккю] сетует:  
\end{ver}

\begin{ver}
  \begin{verses}
    Блеск Дайтё почти полностью загашен.\\
    /Ныне/Нынче/ кто на горе Рёхо [Дайтоку-джи] знает что-нибудь о
    нём?\\
    Через тысячу лет, только [духовные] потомки Токая [Иккю],\\
    Будут [упорно] стараться, чтобы поддерживать его дух в живых.
  \end{verses}
\end{ver}
\begin{ver}[1]\it
Каллиграфия кисти Иккю, обозначающая ``Иккю'' -- ``Одна Пауза.'' (Музей
искуств Фуджита, Япония)
\end{ver}

\begin{ver}[1]
Oтдельная горизонтальная линия обозначающая ``один''
также служит фаллическим символом, в соответстсвии с тем, что секс
это существенный элемент Дзен стиля Иккю. В стихотворении над
своим именем Иккю сетует:  
\end{ver}

\begin{ver}[1]
  \begin{verses}
    Блеск Дайтё почти иссяк.\\
    Кто нынче на горе Рюхо знает что-нибудь о
    нём?\\
    Через тысячу лет, лишь потомки Токая [Иккю],\\
    Будут бороться, чтоб поддержать его дух.
  \end{verses}

(Гора Рюхо -- местоположение монастыря Дайтоку-дзи, --РМ)
\end{ver}

\begin{ver}[2]
  \begin{verses}
    Блеск Дайтё почти иссяк.\\
    Кто нынче на горе Рюхо знает что-нибудь о
    нём?\\
    Через тысячу лет, лишь потомки Токая,\\
    Будут стараться поддержать его дух.
  \end{verses}

\noindent(Гора Рюхо -- местоположение монастыря Дайтоку-дзи;\\
Токай -- Иккю.\\
--РМ)
\end{ver}
\sep

\subsubsection{Страница 28}
\begin{ver}
Мотто Иккю: ``Войти в /реалм/сферу/ Будды легко, войти в реалм дьявола
сложно'' (Музей Искуств Окаяма, Япония)
\end{ver}

\begin{ver}
Смысл этого мотто, что каждый может быть святым, окружённый Буддами,
но настоящая задача заключается в том, чтобы достич
/просветления/пробуждения/ посреди житейской суматохи, страдания и страстей.  
\end{ver}

\begin{ver}[1]\it
Мотто Иккю: ``Войти в мир Будды легко, войти в мир дьявола
сложно.'' (Музей Искуств Окаяма, Япония)
\end{ver}

\begin{ver}[1]
Смысл этого высказывания, что каждый может быть святым, окружённый Буддами,
но настоящая задача заключается в том, чтобы достичь
просветления среди житейской суматохи, страдания и страстей.  
\end{ver}

\begin{ver}[2]
Смысл этого высказывания, что легко быть святым, будучи окружённым Буддами,
но настоящая задача заключается в том, чтобы достичь
просветления среди житейской суматохи, страдания и страстей.  
\end{ver}
\sep

\subsubsection{Страница 34}
\begin{ver}
Рисунок изображающий эксцентричного китайского монаха по имени Чин с надписью
сделанной Иккю. (Частное собрание, Япония)
\end{ver}

\begin{ver}
/Мятежный/бунтарский/\orig{iconoclastic} Чин отказывался иметь дело с
реглигиозными или светскими властями, влача скудное существование,
изготовляя соломенные сандалии, в результате, получивший прозвище
``Чин -- соломенный сандаль''. Сам рисунок не подписан, возможно он
также принадлежит кисти Иккю. Надпись Иккю восхваляет Чина как мастера
следующего своему сердцу:
\end{ver}

\begin{ver}
  \begin{verses}
    Чай из хвороста и жидкая каша едва /отделяют/не
    допускают/\orig{keep out} его от голода,\\
    Когда он сидит среди /несущихся/кружащихся/\orig{rush} осенних
    листьев.\\
    Он грустит о духовных недостатках молодых монахов,\\
    Его /удовольствия/радость/ это встречать утренний мороз одетым в
    изношенную робу.
  \end{verses}
\end{ver}

\begin{ver}[1]\it
Рисунок изображающий эксцентричного китайского монаха по имени Чин с надписью
сделанной Иккю. (Частное собрание, Япония)
\end{ver}

\begin{ver}[1]
Нонконформистский Чин отказывался иметь дело с
религиозными или светскими властями, и влачил скудное существование,
изготовляя соломенные сандалии, и получил, в результате, прозвище
``Чин -- соломенный сандаль.'' Сам рисунок не подписан, возможно он
также принадлежит кисти Иккю. Надпись Иккю восхваляет Чина как мастера,
следующего своему сердцу:
\end{ver}

\begin{ver}[1]
  \begin{verses}
    Чай из травы и жидкая каша едва не дают ему голодать,\\
    Когда он сидит среди несущихся осенних листьев.\\
    Он грустит о духовных недостатках молодых монахов,\\
    Его радость это встречать утренний мороз одетым в
    изношенную рясу.
  \end{verses}
\end{ver}
\begin{ver}[2]
  \begin{verses}
    Лишь чай из травы и жидкая каша не дают ему голодать,\\
    Когда он сидит среди несущихся осенних листьев.\\
    Он грустит о духовных недостатках молодых монахов,\\
    Его радость это встречать утренний мороз в
    изношенной рясе.
  \end{verses}
\end{ver}
\sep

\subsubsection{Страница 48}
\begin{ver}{}
[Рисунок изображающий] Дарума (Бодхидхарма) ученика Иккю Боккея с
надписью [сделанной] Иккю (Музей Искуств Окаяма, Япония).
\end{ver}

\begin{ver}
/Скромный/экономный/\orig{spare},
/аскетичный/строгий/простой/чистый/незатейливый/\orig{austere},
/интенсивный/сильный/напряженный/\orig{intense} манера\orig{brushwork}
/этого рисунка/ хорошо передаёт Дзен стиля Иккю. Подпись
/отсылает/\orig{refer} к истории Эка (Хуэйкэ), [который] простоял
несколько дней в снегу у пещеры Дарумы, чтобы доказать свою исскренность 
\end{ver}

\begin{ver}
  \begin{verses}
    Кто же ещё сможет успокоить его ум?\\
    Давным-давно стоял Шинко [Эка] перед Шорин-джи,\\
    Не замечая снега /накапливающегося/громоздящегося/ вокруг него\\
    Пока Дарума [сидел] лицом к стене, не открывая своего лица.
  \end{verses}
\end{ver}

\begin{ver}[1]\it
Рисунок изображающий Даруму (Бодхидхарму) выполненный учеником Иккю Боккеем
и надписанный Иккю. (Музей Искуств Окаяма, Япония).
\end{ver}

\begin{ver}[1]
Строгая, аскетичная и интенсивная манера этого рисунка хорошо
передаёт Дзен стиля Иккю. Подпись отсылает к истории Эка (Хуэйкэ),
который, в доказательство своей искренности, простоял несколько дней
в снегу у пещеры, где медитировал Дарума.
\end{ver}

\begin{ver}[1]
  \begin{verses}
    Кто же ещё успокоит его ум?\\
    Давным-давно стоял Эка перед Сорин-дзи,\\
    Заметённый снегом по грудь,\\
    Пока Дарума сидел обращённый к стене, не открывая  лица.
  \end{verses}
\end{ver}     

\begin{ver}[2]
  \begin{verses}
    Кто ещё успокоит ум?\\
    В сугробе по грудь стоял Эка перед пещерой,\\
    Не замечая снега вокруг,\\
    Пока Дарума сидел обращённый к стене, не открывая лица.
  \end{verses}
\end{ver}     

\begin{ver}[1]
Хуэйкэ (яп. Тайсо Эка) -- второй патриарх чань-буддизма.
По одной из легенд, Хуэйкэ обратился к Бодхидхарме: ``Мой ум
неугомонен. Пожалуйста, успокой мой ум!''

-- Принеси мне твой ум, чтобы я мог успокоить его. 

-- Я искал его везде и не мог найти.

-- Вот! -- воскликнул Бодхидхарма, -- Я успокоил твой ум!

С этими словами на Хуэйкэ снизошло просветление.\\
(--РМ)
\end{ver}
\sep

\subsubsection{Страница 53}
\begin{ver}
Иккю и его любовь/любовница Леди Мори (Музей искуств Масаки, Япония)
\end{ver}

\begin{ver}
Над свим портретом Иккю написал
\end{ver}

\begin{ver}
  \begin{verses}
    Внутри этого Дзенского Круга, показано\orig{revealed} всё моё
    тело.\\ 
    Это /и в самом деле/\orig{really} рисунок [изображаящий]
    [ре]инкарнацию Кидо.\\
    Моя слепая министрель поёт о любви, и заставляет этого старого
    /плута/каналью/шельму/негодяя/ улыбаться --\\
    Одна /нота/напев/мелодия/мотив/\orig{tune} с ней под
    /цветами/цветущим [деревом]/ это как десять тысяч
    /вёсен/ручьёв/\orig{springs}.  
  \end{verses}
\end{ver}

\begin{ver}
    Рядом с изибражением Леди Мори, Иккю набросал\orig{brushed} её
    /горькое/живое/мучительное/\orig{poignant} стихотворение
    описывающее их ``с Мая по Декабрь'' /роман/любовную историю/:
\end{ver}

\begin{ver}
  \begin{verses}
    В промежутке между\\
    Глубоким сном и легкой дремотой\orig{deep dream
    and light sleep}\\
    Я всплываю и погружаюсь --\\
    Нельзя /удержать/остановить/\\
    Мой поток горько-сладких слёз.
  \end{verses}
\end{ver}

\begin{ver}[1]\it
Иккю и его пассия Леди Мори. (Музей искуств Масаки, Япония)
\end{ver}

\begin{ver}[1]
  Над своим портретом Иккю написал такое стихотворение:
\end{ver}

\begin{ver}[1]
  \begin{verses}
    Я здесь весь, внутри этого Дзенского Круга.\\ 
    В самом деле это рисунок инкарнации Кидо.\\
    Моя слепая министрель поёт о любви, заставляя этого старого
    плута улыбаться~--\\
    Одна нота с ней под цветущим деревом, как журчание десяти тысяч
    ручьёв.  
  \end{verses}
\end{ver}

\begin{ver}[2]
  \begin{verses}
    Я весь внутри этого Дзенского Круга.\\ 
    Это рисунок настоящей инкарнации Кидо.\\
    Моя слепая министрель поёт о любви, заставляя этого старого
    плута улыбаться~--\\
    Одна нота с ней под цветущим деревом, как журчание десяти тысяч
    ручьёв.  
  \end{verses}
\end{ver}

\begin{ver}[3]
  \begin{verses}
    Здесь я весь, внутри этого Дзенского Круга.\\ 
    Это рисунок настоящей инкарнации Кидо.\\
    Моя слепая министрель поёт о любви, заставляя этого старого
    плута улыбаться~--\\
    Одна нота с ней под цветущим деревом, как журчание десяти тысяч
    ручьёв.  
  \end{verses}
\end{ver}

\begin{ver}[4]
  \begin{verses}
    Внутри этого Дзенского Круга, весь я.\\ 
    Это рисунок настоящей инкарнации Кидо.\\
    Моя слепая министрель поёт о любви, заставляя этого старого
    плута улыбаться~--\\
    Одна нота с ней под цветущим деревом, как журчание десяти тысяч
    ручьёв.  
  \end{verses}
\end{ver}

\begin{ver}[1]
    Рядом с изображением Леди Мори, Иккю набросал её
    горькое стихотворение
    описывающее их ``с Мая по Декабрь'' роман:
\end{ver}

\begin{ver}[1]
  \begin{verses}\it
    В промежутке между\\
    Глубоким сном и легкой дрёмой,\\
    Я всплываю и погружаюсь --\\
    Нельзя удержать\\
    Мой поток горько-сладких слёз.
  \end{verses}
\end{ver}

\begin{ver}[2]
  \begin{verses}
    Между\\
    Сном и дрёмой,\\
    Я всплываю и погружаюсь\\
    И не могу удержать\\
    Поток моих горько-сладких слёз.
  \end{verses}
\end{ver}
 
\begin{ver}[3]
  \begin{verses}
    Между\\
    Дрёмой и сном,\\
    Я всплываю и погружаюсь --\\
    Нельзя удержать\\
    Поток моих горько-сладких слёз.
  \end{verses}
\end{ver}
\sep
 
\subsubsection{Страница 74}
\begin{ver}
Автопортрет /Хакуина в возрасте семидесяти одного
года/семидесятиоднолетнего Хакуина/ (Сокровищница
[храма/монастыря] Шоин-джи, Япония)

/[На этом портрете]/Здесь/ Хакуин показывает свою
/суровую/строгую/твёрдую/непреклонную/\orig{stern} сторону,
``смотрящий свирепо как тигр, и движушийся как бык'', и он кажется
готовым применить свою палку к любому /нерадивому/ленивому/ монаху,
который появится [перед ним]. Другие [его] автопортреты менее
сдержаны, где Хакуин изображает себя в виде косого
болвана\orig{dunce}, /скрытного/осторожного/осмотрительного/ старого
монаха, или как доброго\orig{gentle} дедушку.

Длинная подпись/надпись/ принадлежит ученику Хакуина, Торею, а
каллиграфия над головой мастера воспроизводит
/стих/четверостишие/\orig{verse}, часто использовавшееся Хакуином /по
отношению к себе/\orig{to describe himself}. 
\end{ver}

\begin{ver}
  \begin{verses}
    В /собрании/обществе/\orig{assembly} Будд, /ни один Будда не
    испытывает к нему приязни/все Будды его не любят/,\\
    В скоплении\orig{congregation} демонов, все демоны /питают к нему
    отвращение/ненавидят его/\\ 
    Этого дряхлого старого /лысый/плешивый человек/\orig{baldy} опять
    появляется здесь изображённым на бумаге.
  \end{verses}
\end{ver}

\begin{ver}[1]\it
Автопортрет семидесятиоднолетнего Хакуинa. (Сокровищница
храма Сооин-дзи, Япония)
\end{ver}

\begin{ver}[1]
На этом автопортрете Хакуин показывает свою суровую сторону характера,
``смотрящий свирепо как тигр, и движушийся как бык'', и он кажется
готовым применить свою палку к любому нерадивому монаху, который
появится перед ним. Другие его автопортреты не так серьёзны , на них
Хакуин изображает себя в виде косого болвана, осмотрительного старого
монаха, или как доброго дедушку.
\end{ver}

\begin{ver}[2]
Хакуин показывает суровую сторону своего характера, ``со взглядом
тигра и повадкой быка'', и кажется он готов угостить своей палкой
любого нерадивого монаха, окажись тот перед ним. Другие его
автопортреты менее строгие, на них Хакуин изображает себя в виде
косого болвана, скрытного старого монаха, или как добродушного
старичка.
\end{ver}

\begin{ver}[3]
На этом автопортрете Хакуин -- ``со взглядом тигра и повадкой быка'', --
показывает суровую сторону своего характера, и кажется он готов
угостить своей палкой любого нерадивого монаха, окажись тот перед
ним. Другие его автопортреты менее строгие, на них Хакуин изображает
себя в виде косого болвана, скрытного старого монаха, или как
добродушного старичка.
\end{ver}

\begin{ver}[1]
Длинная надпись принадлежит ученику Хакуина, Торею, а
каллиграфия над головой мастера воспроизводит
стихотворение, часто использовавшееся Хакуином по
отношению к себе. 
\end{ver}

\begin{ver}[1]
  \begin{verses}
    В обществе Будд, ни один не испытывает к нему приязни,\\
    Среди демонов, каждый ненавидит его,\\ 
    Этого дряхлого лысого старика, который опять
    изображён на бумаге!
  \end{verses}
\end{ver}

\begin{ver}[2]
  \begin{verses}
    В обществе Будд, ни один не испытывает к нему приязни,\\
    Среди демонов, каждый ненавидит его,\\ 
    Этого дряхлого лысого старика, вот он опять
    изображён на бумаге!
  \end{verses}
\end{ver}

\begin{ver}[3]
  \begin{verses}
    В обществе Будд, ни один не испытывает к нему приязни,\\
    Среди демонов, каждый ненавидит его,\\ 
    Этого дряхлого лысого старика, снова изображёного на бумаге!
  \end{verses}
\end{ver}
\comm{Плохо}
\sep

\subsubsection{Страница 82}
\begin{ver}
Посох [в виде] дракона с метёлкой\orig{and Fly Whisk=и метёлка против
мух?}, кисти Хакуина. (Коллекция Геншин, США)

Это пример, так называемой ``инка'', или Дзенского диплома, которыми
Хакуин награждал своих учеников. Посох-дракон и метёлка символизируют
просветление и /учительский авторитет/\orig{teaching authority} Дзен.
Подпись гласит:

\begin{quote}
  Осень 1758. Мирянин Сугизаки Йакуцен из Ширако в провинции Исе,
  разбил тяжёлые препятствия, услышав ``Звук одной ладони''. Поэтому я
  начертал\orig{brushed} это, как подтверждение его достижения. Не тот
  ли, кто поставил [на кон] всё для достижения этой великой награды,
  /станет/является/ воистину доблестным воином? 
\end{quote}
\end{ver}

\begin{ver}[1]\it
Посох-Дракон с Кистью, кисти Хакуина. (Коллекция Геншин, США)
\end{ver}

\begin{ver}[1]
Это пример, так называемой ``инка'', или Дзенского диплома, которыми
Хакуин награждал своих учеников. Посох в виде дракона и кисть символизируют
просветление и наставнический авторитет Дзен.
Подпись гласит:
\end{ver}

\begin{ver}[1]
\begin{quote}
  Осень 1758. Мирянин Сугидзаки Якуцен из Сирако в провинции Исэ,
  разбил неприступные преграды, услышав ``Звук одной ладони''. Поэтому я
  начертал это, как подтверждение его достижения. 
  Разве поставивший на кон всё для достижения великой награды не
  является воистину доблестным воином? 
\end{quote}
\end{ver}
\comm{Кисть/Метёлка?}
\sep


\subsubsection{Страница 98}
\begin{ver}
Гигантский Дарума (Бодхидхарма), кисти Хакуина. (Коллекция Мюррей Смит,
США)

Подпись гласит: ``[Дзен] указывает прямо /в/на/ человеческое сердце;
проникни в свою природу и стань Буддой.''
\end{ver}

\begin{ver}[1]\it
Гигантский Дарума (Бодхидхарма), кисти Хакуина. (Коллекция Мюррей Смит,
США)
\end{ver}

\begin{ver}[1]
Подпись гласит: ``Дзен указывает прямо в человеческое сердце;
проникни в свою природу и стань Буддой.''
\end{ver}
\sep

\subsubsection{Страница 99}
\begin{ver}
Богиня сострадания Каннон, кисти Хакуина (Коллекция Геншин, США)

Кроме своих свирепых, мужественных [рисунков] Дарумы, Хакуин был
способен /идентифицироваться/ и с женским [началом] выражающим любовь
и сострадание, и превносил эти качества в /своё искусное Дзенское
искуство/.
Надпись на рисунке говорит: ``Океан счастья и долголетия для всех
чувствующих существ.''
\end{ver}

\begin{ver}[1]\it
Богиня сострадания Каннон, кисти Хакуина. (Коллекция Геншин, США)
\end{ver}

\begin{ver}[1]
Кроме своих свирепых, мужественных рисунков Дарумы, Хакуин ценил и
женское начало, выражающее любовь и сострадание. Эти качества также
были темой его искусных Дзенских работ.
Надпись на рисунке говорит: ``Океан счастья и долголетия для всех
чувствующих существ.''
\end{ver}
\comm{свирепых, мужественных -- плохо}
\sep
 
\subsubsection{Страница 100}
\begin{ver}
Мотто Хакуина: ``Медитация /среди/в/ действии в миллион раз лучше
медитации в покое!'' (Коллекция Танака, Япония)
\end{ver}

\begin{ver}[1]\it
Мотто Хакуина: ``Медитация в действии в миллион раз лучше
медитации в покое!'' (Коллекция Танака, Япония)
\end{ver}
\sep

\subsubsection{Страница 112}
\begin{ver}
Статуя Рёкана. Эта статуя стоит недалеко от могилы Рёкана во владениях
семьи Кимура в Префектуре Ниигата\comm{translit, DONE}
\end{ver}

\begin{ver}[1]\it
Статуя Рёкана. 
\end{ver}

\begin{ver}[1]
Эта статуя находится недалеко от могилы Рёкана во владениях
семьи Кимура в Префектуре Ниигата.
\end{ver}
\sep
 
\subsubsection{Страница 136}
\begin{ver}
Черновик каллиграфии (Частная коллекция, Япония). Всю жизнь Рёкан
изучал калиграфию и сегодня его черновики ценятся /как/в качестве/ шедевров. 
\end{ver}

\begin{ver}[1]\it
Черновик каллиграфии Рёкана. (Частная коллекция, Япония)
\end{ver}

\begin{ver}[1]
Всю жизнь Рёкан изучал калиграфию и сегодня даже его черновики ценятся
в качестве шедевров.
\end{ver}
\sep
 
\subsubsection{Страница 142}
\begin{ver}
Паринирвана Будды, кисти Рёкана. (Частная коллекция, Япония)

Редкий экземпляр Дзенского рисунка кисти Рёкана. Манера письма
бесхитростна и совершенно /непринуждённа/искренна/непосредственна/без
аффектации/ и полна мягкого юмора.
\end{ver}

\begin{ver}[1]\it
Паринирвана Будды, кисти Рёкана. (Частная коллекция, Япония)
\end{ver}
\begin{ver}[1]
Редкий экземпляр Дзенского рисунка кисти Рёкана. Манера письма
бесхитростна, совершенно непринуждённа и полна мягкого юмора.
\end{ver}
\sep
 
\subsubsection{Страница 150}
\begin{ver}
Гого-ан. 

Это копия уединенной хижины Рёкана на горе Кугами, построенная в
начале двадцатого века.
\end{ver}

\begin{ver}[1]\it
Гого-ан. 
\end{ver}

\begin{ver}[1]
Это копия уединенной хижины Рёкана на горе Кугами, построенная в
начале двадцатого века.
\end{ver}
\sep
 
\subsubsection{Страница 151} 
\begin{ver}
Интерьер Гого-ан.

На фотографии видны статуя Рёкана в алкове и рядом Буддистский алтарь.
\end{ver}

\begin{ver}[1]\it
Интерьер Гого-ан.
\end{ver}
\begin{ver}[1]
На фотографии видны статуя Рёкана в алькове и Буддистский алтарь.
\end{ver}
\sep

\subsubsection{Страница 156}
\begin{ver}
Мотто Рёкана: ``Относись с уважением к старшим и с пониманием к
младшим'' (Коллекция Кимура, Япония)

Написано Рёканом в последний год его жизни.
\end{ver}

\begin{ver}[1]\it
Мотто Рёкана: ``Относись с уважением к старшим и с пониманием к
младшим.'' (Коллекция Кимура, Япония)
\end{ver}

\begin{ver}[1]
  Написано Рёканом в последний год его жизни.
\end{ver}
\sep

