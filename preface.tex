\str{7}

\begin{ver} 
{\em Три Мастера Дзен} это /истории
жизней/биографии/\nat{жизнеописания}/ Иккю Содзюна (1394---1481),
Хакуина Экаку (1689---1768) и Рёкана Тайгу (1758---1831).  Каждый из
них /доносил/показывал/выражал/ Дзен своим собственным способом.
Радикальный Иккю, ``Сумашедшее Облако'', был
/нетрадиционен/самобытен/, бескомпромисен и
/воинственен/агресивен/. \comm{или был ``полные прилагательные''
мастером} Он /неустанно/неуклонно/ /нападал
на/выявлял/\nat{критиковал} /фальш/притворство/ханжество/ и лицемерие,
не скрывая ничего о себе, включая свою /личную/сексуальную/ жизнь, что
делает его /уникальным/особенным/ среди других /дзенских
служителей/адептов Дзен/ с /непроницаемыми/бесстрастными/ лицами,
которые особенно хорошо умели скрывать свои эмоции.
\end{ver}

\begin{ver}[1] 
{\em Три Мастера Дзен} это жизнеописания Иккю Содзюна
(1394---1481), Хакуина Экаку (1689---1768) и Рёкана Тайгу
(1758---1831).  Каждый из них демонстрировал Дзен своим
собственным способом. 
 Радикальный Иккю, ``Безумное Облако'', был
самобытен, бескомпромисен и
воинственен. Он неустанно протестовал против окружавших его
фальши, ханжества и лицемерия, не скрывая при этом ничего о себе,
включая свою личную жизнь, что делает его
особенным среди других адептов Дзен
с бесстрастными лицами, так хорошо умевших
скрывать свои эмоции.
\end{ver}
\sep

\begin{ver}
/В то время как/Тогда как/ Иккю осуждал систему дипломов принятую в
оффициальных кругах Дзен \comm{ужасно}, отвергнув инка (диплом), так
называемый ``сертификат просветления'' выданный ему его мастером
\comm{не дословно, см ориг}, и отказываясь выдавать подобного рода
документы /своим/своим собственным/ ученикам, Хакуин выдал больше
дипломов, чем любой другой /подлинный/бона фиде/ мастер /в Японской
истории/в истории Японского Буддизма направления Дзен/ используя эту
систему, чтобы вырастить/взростить /небольшую/маленькую/ армию
последователей \comm{не точно, practitioner - человек практикующий
  какой-то вид деятельности} Дзен, упорно стремящихся
/увидеть/проникнуть/осознать в своё /естество/природу/ и стать
Буддами. \comm{в оригинале сп предложение -- которые...}
\end{ver}

\begin{ver}[1]
В отличие от Иккю, который осуждал систему дипломов принятую в в
утвердившейся системе Дзен \comm{не очень точно}, отвергнув инка
(диплом), так называемый ``сертификат просветления'', выданный ему его
мастером, и отказывался выдавать подобного рода документы своим
собственным ученикам, Хакуин выдал больше дипломов, чем любой другой
настоящий Дзен-мастер в истории Японского Буддизма, используя эту
систему, чтобы вырастить небольшую армию последователей Дзен, упорно
стремящихся осознать свою природу и стать Буддами.
\end{ver}
\comm{establishment -- ??, здесь и в нескольких местах далее.}
\sep

\begin{ver}
Внушительный\orig{big=большой}, /решительный/дерзкий/энергичный/,
динамичный Хакуин \comm{поменять порядок слов} настаивал, что каждый
-- мужчина или женщина, монах или обыватель, крестьянин или аристократ
-- может практиковать ``медитацию в действии.''
\end{ver}

\begin{ver}[1]
Хакуин -- внушительный, энергичный, динамичный -- настаивал, что каждый
-- мужчина или женщина, монах или обыватель, крестьянин или аристократ
-- может практиковать ``медитацию в действии.''
\end{ver}
\sep

\begin{ver}
Рёкан, /в отличие/с другой стороны/\orig{in contrast}, был
/тихим/кротким/мягким/спокойным/, затворническим\orig{reclusive}, скромным
мастером. /Никогда не став/Не будучи/ настоятелем, Рёкан жил тихой и
простой жизнью в скромных\orig{little} хижинах, проводя свои дни, играя с
деревенскими детишками, за чашечкой саке с местными фермерами, общаясь
с природой и /занимаясь поэзией/сочиняя стихи/.
\end{ver}

\begin{ver}[1]
Рёкан, с другой стороны, был мягким, спокойным и скромным
затворником. Никогда не став настоятелем, Рёкан жил тихой и простой
жизнью в скромных хижинах, и проводил свои дни, играя с деревенскими
детишками, за чашечкой саке с местными крестьянами, общаясь с природой
и сочиняя стихи.
\end{ver}
\sep

\begin{ver}
Каждый\comm{см ориг} был /художником/артизаном/\orig{artist} высочайшего класса
использовавшим кисть, чернила и бумагу как средство передачи
учения\orig{Zen teaching}.
\end{ver}

\begin{ver}[1]
Каждый из них был художником высочайшего класса
использовавшим кисть, чернила и бумагу как средство передачи
учения.
\end{ver}
\sep

\begin{ver}
На протяжении веков /великолепное/изумительное/\orig{magnificient}
искуство Иккю, Хакуина и Рёкана -- мазки просветления\comm{ужасно} --
вдохновляло, наставляло и /радовало/доставляло
удовольствие/восторгало/вызывало восхищение/\orig{delighted} поколения
учеников Дзен.
\end{ver}

\begin{ver}[1]
На протяжении веков и до сего времени изумительное искуство Иккю,
Хакуина и Рёкана -- линии просветления\comm{???} --
вдохновляло, наставляло и вызывало восхищение у многих поколений
учеников Дзен.
\end{ver}
\sep

\begin{ver}
Хотя в основном я пользовался первичными источниками [имеющими
  отношение] к биографии Иккю, мне казалось\orig{considered}
необходимым также /включить/рассмотреть/\orig{include} ряд наиболее
правдоподобных анекдотов об этом /исключительном/своеобразном/ монахе,
чтобы /изобразить/нарисовать портрет/ Иккю, лелеемый\orig{cherished} его
многочисленными поклонниками.   
Каждая история, даже если не дословно правивая,  /доносит/выражает/
один из важных принципов Дзен /типичных/характерных/ для Иккю.
Что касается жизней Хакуина и Рёкана, тут у нас твёрдая почва под
ногами, благодаря большому количеству автобиографического материала и
заметок современников этих двух мастеров\comm{не точно}. Все переводы
/принадлежат моему перу/мои собственные/мои/.
\end{ver}

\begin{ver}[1]
В основном я пользовался первичными источниками имеющими
  отношение к биографии Иккю, но мне показалось
необходимым также включить ряд наиболее
правдоподобных анекдотов об этом своеобразном монахе,
чтобы нарисовать портрет Иккю, лелеемый его
многочисленными поклонниками.   
Каждая история, даже если не дословно правдивая, демонстрирует
один из важных принципов Дзен характерных для Иккю.

Что касается жизней Хакуина и Рёкана, то тут почва под ногами более
твердая, благодаря большому количеству автобиографического материала и
заметок современников этих двух мастеров. Все переводы цитат и
стихов мои.
\end{ver}
\sep

\begin{ver}
Дзен одновременно конкретный и универсальный. /Частности/конкретика/
[в книге] {\em Три Мастера Дзен} относятся ко времени месту и
обстоятельствам; /универсальность/общность/ это суть Дзен, раскрытая в
жизнях и учении Иккю, Хакуина и Рёкана. Вместе они они представляют
/вечное/вневременное/ послание ценное для любого человека в любую эпоху.
\end{ver}

\begin{ver}[1]
Дзен одновременно конкретен и универсален. Конкретное в книге {\em Три
Мастера Дзен} относится ко времени, месту и обстоятельствам;
универсальное это суть Дзен, раскрытая в жизнях и учении Иккю,
Хакуина и Рёкана. Вместе они они представляют вневременное послание,
ценное для любого человека в любую эпоху.
\end{ver}

\hfill\parbox{30mm}{Джон Стивенс\\
                    \em Гонолулу, 1992}
\sep
